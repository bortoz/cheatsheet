\RequirePackage[2020-02-02]{latexrelease}
\documentclass[8pt,landscape]{article}
\usepackage[italian]{babel}
\usepackage[utf8]{inputenc}
\usepackage{multicol}
\usepackage{calc}
\usepackage[landscape]{geometry}
\usepackage{hyperref}
\usepackage{amsmath}
\usepackage{amssymb}
\usepackage{tabularx}
\usepackage{caption}
\usepackage{verbatim}
\usepackage{systeme}
\usepackage{nicefrac}
\usepackage{accents}
\usepackage{enumitem}
\usepackage[printwatermark]{xwatermark}
\usepackage{tikz}
\usepackage[compact]{titlesec}
\usepackage{microtype}
\usepackage[flushleft]{threeparttable}
\usepackage{textcomp}
\usepackage{pdflscape}
\usepackage{pifont}
\usepackage{pgfplots}
\usepackage{icomma}
\usepackage{wrapfig}
\usepackage[type={CC}, modifier={by-nc-sa}, version={4.0}]{doclicense}

% Page margins
\geometry{top=0.5cm,left=0.5cm,right=0.5cm,bottom=0.5cm}

% Tikz
\usetikzlibrary{calc,matrix}

% Turn off header and footer
\pagestyle{empty}

% Reduce size of \section e \subsection
\titleformat{\section}{\normalfont\large\bfseries}{\thesection}{1em}{}
\titleformat{\subsection}{\normalfont\normalsize\bfseries}{\thesubsection}{1em}{}
\titlespacing{\section}{0pt}{0ex}{-0.5ex}
\titlespacing{\subsection}{0pt}{0ex}{-0.5ex}

% Define BibTeX command
\def\BibTeX{{\rm B\kern-.05em{\sc i\kern-.025em b}\kern-.08em
		T\kern-.1667em\lower.7ex\hbox{E}\kern-.125emX}}

% Don't print section numbers
\setcounter{secnumdepth}{0}

\setlength{\parindent}{0pt}
\setlength{\parskip}{0pt plus 0.5ex}
\setlist[itemize]{noitemsep, nolistsep}

% \newwatermark[allpages,color=black!10,angle=45,scale=6,xpos=-20,ypos=15]{BOZZA}

\begin{document}

\raggedright
\footnotesize
\begin{multicols}{3}

% multicol parameters
% These lengths are set only within the two main columns
%\setlength{\columnseprule}{0.25pt}
\setlength{\premulticols}{1pt}
\setlength{\postmulticols}{1pt}
\setlength{\multicolsep}{1pt}
\setlength{\columnsep}{2pt}

{\Large{\textbf{Matematica del discreto}}}

% Relazioni

\section{Relazioni}
Una relazione binaria tra $X$ e $Y$ è un insieme di coppie ordinate $(x,y)\in X\times Y$. Una relazione può essere:
\begin{tabular}{ll}
	Riflessiva & $xRx$ \\
	Simmetrica & $xRy \Rightarrow yRx$ \\
	Antisimmetrica debole & $xRy \wedge yRx \Rightarrow x=y$ \\
	Antisimmetrica forte & $xRy \Rightarrow \neg yRx$ \\
	Transitiva & $xRy, yRz \Rightarrow xRz$ \\
    Equivalenza & riflessiva, simmetrica e transitiva \\
    Ordine debole & antisimmetrica debole e transitiva \\
    Ordine forte & antisimmetrica forte e transitiva \\
\end{tabular}


% Congruenze

\section{Equazioni diofantee}
L'equazione $ax + by = d$ ha soluzione se e solo se $d$ è multiplo di $\text{mcd}(a,b)$, le infite soluzioni hanno la forma $(x + kb/d,y - ka/d)$. \\
Per risolvere $ax + by = d$ basta risolvere $ax_1 + by_1 = c = \text{mcd}(a, b)$, $x = x_1d/c, y = y_1d/c$ \\
\section{Congruenze lineari}
$ax \equiv 1 (\text{mod } n)$ ha soluzione se e solo se $a$ è coprimo con $n$, la soluzione è $x \equiv a^{\varphi(n)-1} (\text{mod } n)$.
\section{Teorema del resto cinese}
\begin{tabular}{@{}l@{}l@{}}
$\left\{
\begin{array}{ll}
x \equiv b_1 &(\text{mod } n_1) \\
\vdots \\
x \equiv b_k &(\text{mod } n_1) \\
\end{array}
\right.$ &
$\begin{array}{l}
\vspace{1.2mm}\hspace{5mm}
N_i = \prod_{j \neq i} n_i \\
\vspace{1mm}\hspace{5mm}
N_i y_i \equiv 1 \quad (\text{mod } n_i) \\
\vspace{1mm}\hspace{5mm}
x \equiv \sum b_i y_i N_i \quad (\text{mod } \prod n_i)\\
\end{array}$
\end{tabular}
\section{Phi di Eulero}
$\varphi (p^k) = p^k-p^{k-1}$ \\
$\varphi (n)=\varphi (p_{1}^{k_{1}})\varphi (p_{2}^{k_{2}})\ldots \varphi (p_{r}^{k_{r}})$

% Strutture algebriche

\section{Gruppi}
Data struttura algebrica $G$ e una leggere di composizone $\times$, $(G, \times)$ è un gruppo se: \\
\begin{tabular}{ll}
	Associativa & $(v \times w) \times z = v \times (w + z)$ \\
	Elemento neutro & $v \times 1 = v$ \\
	Elemento inverso & $\exists x \quad v \times x = 1$
\end{tabular}

Un gruppo è abeliano se: $v \times w = w \times v$ \\
\section{Anelli e campi}
$(A,+,\times)$ è un anello se:
\begin{tabular}{l}
$(A, +)$ è un gruppo abeliano; \\
$(A, \times)$ è un monoide; \\
vale la proprietà distributiva del prodotto rispetto alla somma. \\
\end{tabular}
L'insieme degli elementi invertibili di un anello forma un gruppo \\
Un elemento $a \neq 0$ si dice divisore dello zero se $\exists b$ tc $a \times b = 0$. \\
Un campo è un anello abeliano in cui ogni elemento è invertibile. \\
\section{Gruppi di permutazione}
Un gruppo di permutazione è l'insieme delle applicazioni bigettive su un insieme $X$. \\
$\sigma = \begin{pmatrix}
    1 & 2 & \dots & n \\
    \sigma(1) & \sigma(2) & \dots & \sigma(n) \\ 
\end{pmatrix}$
$\sigma = \begin{pmatrix}
    a_1 & \sigma(a_1) & \sigma(\sigma(a_1)) & \ldots
\end{pmatrix}$ \\
Due cicli sono disgiunti (permutano) se operano su insiemi disgiunti. \\
L'ordine di un ciclo è il più piccolo $m$ tale che $\sigma^M = I$. \\
Teorema di Lagrange: l'ordine di un sottogruppo $H$ di un gruppo $G$ è un divisore dell'ordine di $G$. \\
Un sottogruppo ciclico è l'insieme: $\{1, x, x^2, \ldots, x^n\}$. \\
Un gruppo $G$ è ciclico se tutti gli elementi possono essere espressi come potenza di $x \in G$, $x$ è un generatore di $G$. \\
Se un gruppo è ciclico ogni su sottogruppo è ciclico. \\
Se un gruppo è ciclico di ordine $n$, per ogni divisore $d$ di $n$, esiste ed è unico un sottogruppo di ordine $d$. \\


% Matrici

\section{Determinante}
$\det: M_{n,n}(K) \rightarrow K$

\begin{tabularx}{\textwidth}{lX}
	$n = 1$ & $A = [a] \quad \det{A} = a$ \\
	$n > 1$ &
	Ricorsivamente \newline
	$A_{ij}$ ottenuta da $A$ togliendo riga $i$ e colonna $j$ \newline
	$M_{ij} = \det A_{ij}$ (detto minore complementare) \newline
	$C_{ij} = (-1)^{i+j}M_{ij}$ (detto complemento algebrico) \newline
	$\det A = \sum_{i=1}^{n} a_{1i}C_{1i}$ \\
\end{tabularx}

Th di Laplace: si può usare una riga o una colonna qualsiasi.

\begin{tabular}{l}
	$\det A = \det A^T$ \\
	Se una riga o colonna ha tutti zeri: $\det A = 0$ \\
	Se si scambiano 2 righe: $\det A' = -\det A$ \\
	Se due righe sono uguali: $\det A = 0$ \\
	Moltiplicando una riga: $\det A' = k\det A$, $\det kA = k^n\det A$ \\
	Sommando ad una riga un'altra riga: $\det A' = \det A$ \\
	In una matrice triangolare: $\det A = \prod_{i=1}^{n} a_{ii}$ \\
	Teorema di Binet: $\det AB = \det A \cdot \det B$ \\
	$A$ è inveribile se e solo se: $\det A \neq 0$ \\
	Se $A$ è invertibile allora: $\det A^{-1} = (\det A)^{-1}$ \\
\end{tabular}

\section{Sistemi lineari}
\begin{tabular}{@{}l@{}l@{}}
	$\begin{cases}
		a_{11}x_1 + \cdots + a_{1n}x_n = b_1 \\[-0.3em]
		a_{21}x_1 + \cdots + a_{2n}x_n = b_2 \\[-0.3em]
		\vdots \\[-0.3em]
		a_{m1}x_1 + \cdots + a_{mn}x_n = b_m \\
	\end{cases}$ &
	$A|b = \left[
		\arraycolsep=1.7pt\def\arraystretch{1.2}
		\begin{array}{ccc|c}
			a_{11} & \cdots & a_{1n} & b_1 \\[-0.3em]
			a_{21} & \cdots & a_{2n} & b_2 \\[-0.3em]
			\vdots & \ddots & \vdots & \vdots \\[-0.3em]
			a_{m1} & \cdots & a_{mn} & b_m \\
		\end{array}
		\right]$
\end{tabular}

\section{Rango}
Minore di ordine $p$: il determinate di una sottomatrice quadrata di ordine $p$. \\
Rango: il più grande $p$ tale che esista un minore di ordine $p$ non nullo,
è uguale al numero di pivot durante GJ. \\
\section{Autovalori e autovettori}

$v \neq 0$ è autovettore di autovalore $\lambda$ se $Av = \lambda v$ \\

Per trovare gli autovalori risolvere $\det(A-\lambda I) = 0$. \\
Il determinante $p(\lambda)$ viene chiamato \emph{polinomio caratteristico}. \\

Per trovare gli autovalori risolvere $(A-\lambda I)v=0$. \\

Molteciplità algebrica $m_a$: molteciplità di $\lambda$ in $p(\lambda)$. \\
Molteciplità geometrica $m_g$: $\dim (\ker (A-\lambda I))$. \\

\section{Matrici simili}
Due matrici $A$, $B$ sono simili se $\exists M$, $A=MBM^{-1}$, due matrici sono simili se rappresentano uno stesso omomorfismo in basi diverse. \\
Due matrici simili hanno gli stessi autovalori, lo stesso determiante e lo stesso rango. \\


% Spazi vettoriali

\section{Spazi vettoriali}
Uno spazio vettoriale su un campo $\mathbb{K}$ è un insieme $V$ su cui è definita una somma e un prodotto scalare tale che:
\begin{tabular}{l}
    $(V,+)$ è un gruppo abeliano \\
    $k(v+w) = kv+kw$ \\
    $(k_1+k_2)v = k_1v+k_2v$ \\
    $1 \times v = v$ \\
\end{tabular}

$W$ è un sottospazio vettoriale di $V$ se $W$ è chiuso per combinazioni lineari: $k_1w_1 + k_2w_2 \in W$ \\
$\text{Span}(I)$: insieme delle combinazioni lineari di $I$. \\
Uno sv è finitamente generato se $\exists I \subseteq V$ tale che $V=\text{Span}(I)$, $I$ è un insiemi di generatore per $V$. \\
$I$ è linearmente indipendente se esiste un unico modo di generare $0$. \\
Una base è un insieme di generatori linearmente indipendente. \\
Tutte le basi hanno la stessa cardinalita, detta dimensione. \\ 
Th di Grassman: $\dim (S) + \dim (T) = \dim (S \cap T) + \dim (S+T)$ \\


\section{Omomorfismo}
Un omomorfismo è una funzione $f\text{: } V \rightarrow W$ tale che:
$f(hv+kw)=hf(v)+kf(w)$. \\
$\ker f = \{ v \in V \ | \ f(v)=0 \}$ \\
$\text{Im} f = \{ w \in W \ | \ \exists v, w=f(v) \}$ \\
$\ker f$ e $\text{Im} f$ sono sottospazi vettoriali rispettivamente di $V$ e $W$. \\
Se $V$ è generato da $\{ v_1, \ldots \}$ allore $\text{Im} f$ è generato da $\{ f(v_1), \ldots \}$. \\
Teorema nullità più rango: $\dim V = \dim (\ker f) + \dim (\text{Im}f)$. \\
Iniettivo: $f(v)=f(w) \Leftrightarrow v=w$. \\
Surgettivo: $\text{Im} f = W$. \\
Isomorfismo: sia iniettivo che surgettivo. \\

\section{Omomorfismi mediante matrici}
$\mathbb{V} = \{v_1, \ldots, v_n\}$ base di $V$, $\mathbb{W} = \{w_1, \ldots, w_m\}$ base di $W$ \\
\begin{tabular}{@{}l@{}l@{}}
	$\begin{matrix}
		f(v_1) = a_{11}w_1 + \cdots + a_{1m}w_m \\
		\vdots \\
		f(v_n) = a_{11}w_1 + \cdots + a_{nm}w_m \\
	\end{matrix}$ &
	$\rightarrow A = \left[
        \arraycolsep=1.7pt\def\arraystretch{1.2}
        \begin{array}{ccc}
            a_{11} & \cdots & a_{1n} \\
            \vdots & \ddots & \vdots \\
            a_{m1} & \cdots & a_{mn} \\
        \end{array}
    \right]$
\end{tabular} \\
$\dim(\text{Im}f) = \text{rg}(A)$ \\
\section{Diagonalizzazione}
Una matrice si dice diagonalizzabile se è simile ad una matrice diagonale. \\
Una matrice è diagonalizzabile se:

\begin{tabular}{l}
    $\sum m_a(a_i) = n$, non ci sono soluzioni complesse \\
    $m_g(a_i) = m_a(a_i) \Longleftrightarrow  n - \text{rg}(A - a_i I) = m_a(a_i)$ \\
\end{tabular}

% Geometria

\section{Vettori}
Vettore: $\overrightarrow{v} = (v_1, v_1, \ldots, v_n)$ \\
Norma: $||\vec{v}|| = \sqrt{v_1^2 + v_2^2 + ... + v_n^2}$ \\
Prodotto scalare: $\overrightarrow{v} \overrightarrow{w} = \sum_{i=1}^n v_i w_i = ||\overrightarrow{v}|| \cdot ||\overrightarrow{w}|| \cos \theta$ \\
Angolo tra vettori: $\cos{\theta} = \frac{\overrightarrow{v} \cdot \overrightarrow{w}}{||\overrightarrow{v}|| \cdot ||\overrightarrow{w}||}$ \\
Perpendicolari: $\overrightarrow{v} \overrightarrow{w} = 0$ \\
Paralleli: $\overrightarrow{v} = k\overrightarrow{w}$ \\
\section{Rette in 2D}
Forma cartesiana: $r \text{: } ax+by+c=0$ \\
Forma parametica: $r \text{: } P+t\overrightarrow{v}$ \\

Da parametrica a cartesiana:
$\begin{cases}
	x = p_x + t v_x \\
	y = p_y + t v_y \\
\end{cases}$ \\

Da cartesiana a parametrica:  $(0, -c/b) + t (1, -a/b)$

\section{Rette in 3D}

Forma cartesiana: $\begin{cases}
	ax_1 + by_1 + cz_1 + d_1 = 0 \\
	ax_2 + by_2 + cz_2 + d_2 = 0 \\
\end{cases}$ \\
Forma parametica: $r \text{: } P+t\overrightarrow{v}$ \\

Da parametrica a cartesiana:
$\begin{cases}
	x = p_x + t v_x \\
	y = p_y + t v_y \\
	z = p_z + t v_y \\
\end{cases}$ \\

Da cartesiana a parametrica:
$\text{F.C.} \Longrightarrow \begin{cases}
	y = m_1 x + q_1 \\
	z = m_2 x + q_2 \\
\end{cases}
\Longrightarrow r \text{: } \begin{pmatrix}
	0 \\
	q_1 \\
	q_2 \\
\end{pmatrix}
+ t \begin{pmatrix}
	1 \\
	m_1 \\
	m_2 \\
\end{pmatrix}$ \\
\section{Piani in 3D}

Forma cartesiana: $ax + by + cz + d = 0$ \\
Forma parametica: $\pi \text{: } P+t\overrightarrow{v}+s\overrightarrow{w}$ \\


\rule{29em}{0.4pt}

\begin{wrapfigure}[3]{r}{8em}
	\vspace{-1.3em}
	\centering
	\doclicenseImage[imagewidth=7em]
\end{wrapfigure}

Basato sul corso \emph{Matematica del discreto} A.A. 2020/2021 del docente Luperi Baglini Lorenzo. \\

Copyright \copyright \, 2022 Alessandro Bortolin.

\doclicenseText
\end{multicols}
\end{document}
