\section{Gruppi di permutazione}
Un gruppo di permutazione è l'insieme delle applicazioni bigettive su un insieme $X$. \\
$\sigma = \begin{pmatrix}
    1 & 2 & \dots & n \\
    \sigma(1) & \sigma(2) & \dots & \sigma(n) \\ 
\end{pmatrix}$
$\sigma = \begin{pmatrix}
    a_1 & \sigma(a_1) & \sigma(\sigma(a_1)) & \ldots
\end{pmatrix}$ \\
Due cicli sono disgiunti (permutano) se operano su insiemi disgiunti. \\
L'ordine di un ciclo è il più piccolo $m$ tale che $\sigma^M = I$. \\
Teorema di Lagrange: l'ordine di un sottogruppo $H$ di un gruppo $G$ è un divisore dell'ordine di $G$. \\
Un sottogruppo ciclico è l'insieme: $\{1, x, x^2, \ldots, x^n\}$. \\
Un gruppo $G$ è ciclico se tutti gli elementi possono essere espressi come potenza di $x \in G$, $x$ è un generatore di $G$. \\
Se un gruppo è ciclico ogni su sottogruppo è ciclico. \\
Se un gruppo è ciclico di ordine $n$, per ogni divisore $d$ di $n$, esiste ed è unico un sottogruppo di ordine $d$. \\
