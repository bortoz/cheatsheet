\section{Determinante}
$\det: M_{n,n}(K) \rightarrow K$

\begin{tabularx}{\textwidth}{lX}
	$n = 1$ & $A = [a] \quad \det{A} = a$ \\
	$n > 1$ &
	Ricorsivamente \newline
	$A_{ij}$ ottenuta da $A$ togliendo riga $i$ e colonna $j$ \newline
	$M_{ij} = \det A_{ij}$ (detto minore complementare) \newline
	$C_{ij} = (-1)^{i+j}M_{ij}$ (detto complemento algebrico) \newline
	$\det A = \sum_{i=1}^{n} a_{1i}C_{1i}$ \\
\end{tabularx}

Th di Laplace: si può usare una riga o una colonna qualsiasi.

\begin{tabular}{l}
	$\det A = \det A^T$ \\
	Se una riga o colonna ha tutti zeri: $\det A = 0$ \\
	Se si scambiano 2 righe: $\det A' = -\det A$ \\
	Se due righe sono uguali: $\det A = 0$ \\
	Moltiplicando una riga: $\det A' = k\det A$, $\det kA = k^n\det A$ \\
	Sommando ad una riga un'altra riga: $\det A' = \det A$ \\
	In una matrice triangolare: $\det A = \prod_{i=1}^{n} a_{ii}$ \\
	Teorema di Binet: $\det AB = \det A \cdot \det B$ \\
	$A$ è inveribile se e solo se: $\det A \neq 0$ \\
	Se $A$ è invertibile allora: $\det A^{-1} = (\det A)^{-1}$ \\
\end{tabular}
