\section{Spazi vettoriali}
Uno spazio vettoriale su un campo $\mathbb{K}$ è un insieme $V$ su cui è definita una somma e un prodotto scalare tale che:
\begin{tabular}{l}
    $(V,+)$ è un gruppo abeliano \\
    $k(v+w) = kv+kw$ \\
    $(k_1+k_2)v = k_1v+k_2v$ \\
    $1 \times v = v$ \\
\end{tabular}

$W$ è un sottospazio vettoriale di $V$ se $W$ è chiuso per combinazioni lineari: $k_1w_1 + k_2w_2 \in W$ \\
$\text{Span}(I)$: insieme delle combinazioni lineari di $I$. \\
Uno sv è finitamente generato se $\exists I \subseteq V$ tale che $V=\text{Span}(I)$, $I$ è un insiemi di generatore per $V$. \\
$I$ è linearmente indipendente se esiste un unico modo di generare $0$. \\
Una base è un insieme di generatori linearmente indipendente. \\
Tutte le basi hanno la stessa cardinalita, detta dimensione. \\ 
Th di Grassman: $\dim (S) + \dim (T) = \dim (S \cap T) + \dim (S+T)$ \\

