\section{Omomorfismo}
Un omomorfismo è una funzione $f\text{: } V \rightarrow W$ tale che:
$f(hv+kw)=hf(v)+kf(w)$. \\
$\ker f = \{ v \in V \ | \ f(v)=0 \}$ \\
$\text{Im} f = \{ w \in W \ | \ \exists v, w=f(v) \}$ \\
$\ker f$ e $\text{Im} f$ sono sottospazi vettoriali rispettivamente di $V$ e $W$. \\
Se $V$ è generato da $\{ v_1, \ldots \}$ allore $\text{Im} f$ è generato da $\{ f(v_1), \ldots \}$. \\
Teorema nullità più rango: $\dim V = \dim (\ker f) + \dim (\text{Im}f)$. \\
Iniettivo: $f(v)=f(w) \Leftrightarrow v=w$. \\
Surgettivo: $\text{Im} f = W$. \\
Isomorfismo: sia iniettivo che surgettivo. \\
