\RequirePackage[2020-02-02]{latexrelease}
\documentclass[10pt,landscape]{article}
\usepackage[italian]{babel}
\usepackage[utf8]{inputenc}
\usepackage{multicol}
\usepackage{calc}
\usepackage[landscape]{geometry}
\usepackage{hyperref}
\usepackage{amsmath}
\usepackage{amssymb}
\usepackage{tabularx}
\usepackage{caption}
\usepackage{verbatim}
\usepackage{systeme}
\usepackage{nicefrac}
\usepackage{accents}
\usepackage{enumitem}
\usepackage[printwatermark]{xwatermark}
\usepackage{tikz}
\usepackage[compact]{titlesec}
\usepackage{microtype}
\usepackage[flushleft]{threeparttable}
\usepackage{textcomp}
\usepackage{pdflscape}
\usepackage{pifont}
\usepackage{pgfplots}
\usepackage{icomma}
\usepackage{wrapfig}
\usepackage[type={CC}, modifier={by-nc-sa}, version={4.0}]{doclicense}

% Page margins
\geometry{top=0.5cm,left=0.5cm,right=0.5cm,bottom=0.5cm}

% Tikz
\usetikzlibrary{calc,matrix}

% Turn off header and footer
\pagestyle{empty}

% Reduce size of \section e \subsection
\titleformat{\section}{\normalfont\large\bfseries}{\thesection}{1em}{}
\titleformat{\subsection}{\normalfont\normalsize\bfseries}{\thesubsection}{1em}{}
% \titlespacing{\section}{0pt}{0ex}{-0.5ex}
% \titlespacing{\subsection}{0pt}{0ex}{-0.5ex}

% Define BibTeX command
\def\BibTeX{{\rm B\kern-.05em{\sc i\kern-.025em b}\kern-.08em
		T\kern-.1667em\lower.7ex\hbox{E}\kern-.125emX}}

% Don't print section numbers
\setcounter{secnumdepth}{0}

\setlength{\parindent}{0pt}
\setlength{\parskip}{0pt plus 0.5ex}
\setlist[itemize]{noitemsep, nolistsep}

% \newwatermark[allpages,color=black!10,angle=45,scale=6,xpos=-20,ypos=15]{BOZZA}

\begin{document}

\raggedright
\footnotesize
\begin{multicols}{3}

% multicol parameters
% These lengths are set only within the two main columns
%\setlength{\columnseprule}{0.25pt}
\setlength{\premulticols}{1pt}
\setlength{\postmulticols}{1pt}
\setlength{\multicolsep}{1pt}
\setlength{\columnsep}{2pt}

{\Large{\textbf{Statistica e analisi dati}}}

\section{Nozioni base di probabilità}
$P(A) + P(\neg A) = 1$ \\
$P(A \cup B) + P(A \cap B) = P(A) + P(B)$ \\
$P(A \cap B) = P(A) \cdot P(B | A)$ \\
$P(A \cap B) = P(A) \cdot P(B)$ se $A$ e $B$ sono indipendenti. \\
$P(B) = \sum_{i=1}^n P(A_i) P(B | A_i)$ \\

\subsection{Teorema di Bayes}
$P(A | B) = \frac{P(A) P(B | A)}{P(B)} = \frac{P(A \cap B)}{P(B)}$ \\

\section{Nozioni base di statistica}

Momento sul discreto: $\left\langle x^r \right\rangle = \frac{1}{n} \sum_{i=1}^n x_i^r$ \\
Momento sul continuo: $\left\langle x^r \right\rangle = \int_{-\infty}^{+\infty} x^rf(x) \, dx$ \\
Media/valore atteso: $\mu = \bar{x} = \left\langle x \right\rangle$ \\
Mediana: $F(x) = \frac{1}{2}$ \\
Primo quartile: $F(x) = \frac{1}{4}$ \\
Terzo quartile: $F(x) = \frac{3}{4}$ \\
Intervallo interquartile: $\Delta = x_{(3\text{° quartile})} - x_{(1\text{° quartile})}$ \\
Varianza: $\sigma^2 = \left\langle \left( x - \bar{x} \right)^2 \right\rangle = \left\langle x^2 \right\rangle - \left\langle x \right\rangle^2$ \\
Deviazione standard: $\sigma = \sqrt{\sigma^2}$ \\
Momento standard: $\mu_r = \frac{\left\langle \left( x - \bar{x} \right)^r \right\rangle}{\sigma^r}$ \\
% TODO: momento centrale?
Skweness: $\mu_3$ \\
Curtosi: $\mu_4$ \\
Funzione di fallibilità: $F(x) = \int_{-\infty}^{x} f(t)dt$ \\
Funzione di sopravvivenza: $S(x) = 1 - F(x)$ \\

\section{Teorema di Chebyshev}
In un intervallo entro due volte la deviazione standard dalla media, è contenuto almento il $75\%$ della probabilità. \\

\section{Distribuzioni}

\subsection{Distribuzione uniforme}
Funzione di densità: $f(t) = \begin{cases}
	\frac{1}{b - a} & \text{se } a \le t \le b \\
	0 & \text{altrimenti} \\
\end{cases}$ \\
Funzione cumulativa: $F(t) = \begin{cases}
	0 & \text{se } t < a \\
	\frac{x - a}{b - a} & \text{se } a \le t \le b \\
	1 & \text{se } t > b \\
\end{cases}$ \\
Media/valore atteso: $\frac{a + b}{2}$ \\
Mediana: $\frac{a + b}{2}$ \\
Varianza: $\frac{(b - a)^2}{12}$ \\

\subsection{Distribuzione geometrica}
La distribuzione geometrica esprime la probabilità che occorra attendere esattamente $i$ tentativi per avere il primo successo.
La distribuzione geometrica è \textbf{senza memoria}. \\

\vspace{1em}

Funzione di densità: $\mathcal{G}(i \, | \, p) = pq^{i-1}$ \\
Funzione cumulativa: $1 - q^i$ \\
Media/valore atteso: $\frac{1}{p}$ \\
Moda: $1$ \\
Varianza: $\frac{q}{p^2}$ \\

\subsection{Distribuzione binomiale}
La distribuzione binomiale esprime la probabilità di avere esattamente $k$ successi su $n$ tentativi. \\

\vspace{1em}

Funzione di densità: $\mathcal{B}(k \, | \, p, n) = \binom{n}{k}p^{k}q^{n-k}$ \\
Media/valore atteso: $np$ \\
Mediana: $\lfloor np \rfloor$ o $\lceil np \rceil$ \\
Moda: $\lfloor (n+1)p \rfloor$ o $\lceil (n+1)p \rceil -1$ \\
Varianza: $npq$ \\

\subsection{Distribuzione esponenziale}
La distribuzione esponenziale esprime la probabilità di attendere esattamente un tempo $t$ per avere il primo evento. \\
La distribuzione esponenziale è \textbf{senza memoria}. \\

\vspace{1em}

Funzione di densità: $f(t \, | \, \lambda) = \lambda e^{-\lambda t}$ \\
Funzione cumulativa: $1 - e^{-\lambda t}$ \\
Media/valore atteso: $\frac{1}{\lambda}$ \\
Mediana: $\frac{\text{ln} 2}{\lambda}$ \\
Moda: $0$ \\
Varianza: $\frac{1}{\lambda^2}$ \\

\subsection{Distribuzione di Poisson}
La distribuzione di Poisson esprime la probabilità di avere esattamente $k$ eventi in un intervallo di tempo quando la media di eventi è $\mu$. \\
La distribuzione di Poisson viene anche usata per approssimare la distribuzione binomiale quando $n$ è molto grande e $p$ molto piccolo. \\
Data una distribuzione esponenziale, la relativa distribuzione di conteggio è una distribuzione di Poisson dove $\mu = \lambda \Delta t$.

\vspace{1em}

Funzione di densità: $\mathcal{P}(k \, | \, \mu = np) = \frac{\mu ^{k}}{k!} e^{-\mu}$ \\
Media/valore atteso: $\mu$ \\
Moda: $\lceil \mu \rceil -1$ e $\lfloor \mu \rfloor$ \\
Varianza: $\mu$ \\
Merge: ovrapponendo due processi Poissoniani con rate $\lambda_1$ e $\lambda_2$, ottengo un processo Poissoniano di rate $\lambda$. \\
Split: dato un processo Poissoniano di rate $\lambda$, estraendo ogni evento con probabilità $p$, ottengo due processi Poissoniani di rate $p\lambda$ e $(1 - p)\lambda$. \\

\subsection{Distribuzione normale (Gaussiana)}
La distribuzione di Poisson viene usata per approssimare la distribuzione binomiale quando $n$ è molto grande e $p$ è ``lontato'' da $0$ e $1$.

\vspace{1em}

Funzione di densità: $\mathcal{N}(x \, | \, \mu, \sigma) = \frac{1}{\sqrt{2\pi \sigma ^{2}}} e^{-{\frac{1}{2}}\frac{(x - \mu)^2}{\sigma^2}}$ \\
Media/valore atteso: $\mu$ \\
Mediana: $\mu$ \\
Moda: $\mu$ \\
Varianza: $\sigma^2$ \\
Standardizzazione: $\mathcal{N}(x \, | \, \mu, \sigma) = \mathcal{N}(z \, | \, 0, 1)$, per $z = \frac{x - \mu}{\sigma}$ \\
Legge tre sigma:
\begin{itemize}
	\item $P(\mu - 1\sigma \le X \le \mu + 1\sigma) \approx 68,27\%$
	\item $P(\mu - 2\sigma \le X \le \mu + 2\sigma) \approx 95,45\%$
	\item $P(\mu - 3\sigma \le X \le \mu + 3\sigma) \approx 99,73\%$
\end{itemize}

\section{Distribuzione ipergeometrica}
La distribuzione ipergeometrica esprime la probabilità di estrarre senza reinserimento $g$ palline vincenti su $n$ estratte da un'urna contenente $G$ palline vincenti e $B$ palline perdenti.

\vspace{1em}

Funzione di densità: $\displaystyle \mathcal{H}(g \, | \, n, G, B) = \frac{\binom{G}{g}\binom{B}{n - g}}{\binom{G + B}{n}}$ \\

\section{Somma di variabili aleatorie}
Media: $\mu_Z = \mu_X + \mu_Y$ \\
Varianza: $\sigma_Z^2 = \sigma_X^2 + \sigma_Y^2$ se $X$ e $Y$ sono indipendenti \\

\section{Distribuzioni campionarie}
Minimo campionario: $f_{\text{min}}(t) = n f(t) \left ( S \left( t \right) \right)^{n - 1}$ \\
Massimo campionario: $f_{\text{max}}(t) = n f(t) \left ( F \left( t \right) \right)^{n - 1}$ \\
Media campionaria: una distribuzione di media $\mu$ e varianza $\frac{\sigma^2}{n}$ \\

\section{Teorema del limite centrale}
Sommando variabili aleatorie indipendenti con distribuzioni qualsiasi, purché dotate di varianza finita, ottengo, nel limite, una variabile Gaussiana: $f_{\text{avg}}(t) = \mathcal{N}\left( t \, | \, \mu, \frac{\sigma^2}{n} \right)$ \\

\section{Correlazione}
Codevianza: $\operatorname{cod}(x, y) = \sum_{i=1}^n (x_i - \mu_x)(y_i - \mu_y)$ \\
Covarianza: $\operatorname{cov}(x, y) = \frac{\operatorname{cod}(x, y)}{n}$ \\
Coefficiente di Pearson: $\rho_{x, y} = \frac{\operatorname{cov}(x, y)}{\sigma_x \sigma_y}$ \\
\begin{itemize}
	\item $\rho < 0$: correlazione negativa;
	\item $\rho = 0$: nessuna correlazione;
	\item $\rho > 0$: correlazione positiva;
	\item $0\phantom{,0} \le |\rho| < 0,3$: correlazione debole;
	\item $0,3 \le |\rho| < 0,7$: correlazione moderata;
	\item $0,7 \le |\rho| < 1\phantom{,0}$: correlazione forte;
	\item $|\rho| = 1$: correlazione perfetta;
\end{itemize}
Coefficiente di Spearman: $r_s = \rho_{\operatorname{R}(X),\operatorname{R}(Y)} = \frac{\operatorname{cov}(\operatorname{R}(X),\operatorname{R}(Y))}{\sigma_{\operatorname{R}(X)}\sigma_{\operatorname{R}(Y)}}$ \\
Dove $R(x)$ è il rango di $x$, ovvero la posizione di $x$ all'intero di $X$. \\

\section{Stima della media}
Dato un campione di $n$ eventi indipendenti da una popolazione con varianza finita $\sigma^2$ e media empirica $m$, la media ``vera'' è distribuita secondo una distribuzione gaussiana con media $m$ e deviazione standard $\frac{\sigma}{\sqrt{n}}$. \\
Con varianza non nota si usa la varianza empirica: $s^2 = \frac{\sum (x_i - m)^2}{n - 1}$ \\
Media stimata: $\mu_{(\text{stima})} = m \pm \frac{\sigma}{\sqrt{n}}$ \\
Intervallo di confidenza del $68\%$: $\phantom{,3} [m - \phantom{1}\frac{\sigma}{\sqrt{n}} \, ; \, m + \phantom{1}\frac{\sigma}{\sqrt{n}}]$ \\
Intervallo di confidenza del $95\%$: $\phantom{,5} [m - 2\frac{\sigma}{\sqrt{n}} \, ; \, m + 2\frac{\sigma}{\sqrt{n}}]$ \\
Intervallo di confidenza del $99,7\%$: $[m - 3\frac{\sigma}{\sqrt{n}} \, ; \, m + 3\frac{\sigma}{\sqrt{n}}]$ \\

Stima della probabilità in una dist. bernulliana: $p_{(\text{stima})} = \frac{k}{n} \pm \frac{\sqrt{k}}{n}$ \\

\vspace{0.3em}

\rule{29em}{0.4pt}

\vspace{0.7em}

\begin{wrapfigure}[3]{r}{8em}
	\vspace{-1.5em}
	\centering
	\doclicenseImage[imagewidth=7em]
\end{wrapfigure}

Basato sul corso \emph{Statistica e analisi dei dati} A.A. 2021/2022 del docente Gianini Gabriele. \\

Copyright \copyright \, 2022 Alessandro Bortolin.

\doclicenseText

\end{multicols}

\pagebreak

\begin{landscape}
	{\Huge $\phantom{}$ \break \break \break}

	\centering
	{\huge Tavola della distribuzione normale standardizzata} \break

	\centering
	{\large $\mathcal{N}(x \, | \, \mu = 0, \sigma = 1)$}

	\centering
	\begin{tikzpicture}[x=0.6pt,y=0.6pt]
    \definecolor[named]{drawColor}{rgb}{0.00,0.00,0.00}
    \definecolor[named]{fillColor}{rgb}{1.00,1.00,1.00}
    \fill[color=fillColor,fill opacity=0.00,] (0,0) rectangle (578.16,289.08);
    
    \begin{scope}
        \path[clip] ( 49.20, 61.20) rectangle (552.96,239.88);
        \definecolor[named]{drawColor}{rgb}{1.00,1.00,1.00}

        \draw[color=drawColor,line cap=round,line join=round,fill opacity=0.00,] ( 67.86, 67.82) -- ( 68.64, 67.87) -- ( 69.41, 67.93) -- ( 70.19, 67.99) -- ( 70.97, 68.05) -- ( 71.74, 68.12) -- ( 72.52, 68.18) -- ( 73.30, 68.25) -- ( 74.08, 68.31) -- ( 74.85, 68.38) -- ( 75.63, 68.46) -- ( 76.41, 68.53) -- ( 77.19, 68.60) -- ( 77.96, 68.68) -- ( 78.74, 68.76) -- ( 79.52, 68.84) -- ( 80.30, 68.92) -- ( 81.07, 69.01) -- ( 81.85, 69.10) -- ( 82.63, 69.19) -- ( 83.41, 69.28) -- ( 84.18, 69.37) -- ( 84.96, 69.47) -- ( 85.74, 69.57) -- ( 86.52, 69.67) -- ( 87.29, 69.77) -- ( 88.07, 69.88) -- ( 88.85, 69.99) -- ( 89.63, 70.10) -- ( 90.40, 70.21) -- ( 91.18, 70.33) -- ( 91.96, 70.45) -- ( 92.73, 70.57) -- ( 93.51, 70.70) -- ( 94.29, 70.82) -- ( 95.07, 70.95) -- ( 95.84, 71.09) -- ( 96.62, 71.23) -- ( 97.40, 71.37) -- ( 98.18, 71.51) -- ( 98.95, 71.66) -- ( 99.73, 71.81) -- (100.51, 71.96) -- (101.29, 72.11) -- (102.06, 72.27) -- (102.84, 72.44) -- (103.62, 72.61) -- (104.40, 72.78) -- (105.17, 72.95) -- (105.95, 73.13) -- (106.73, 73.31) -- (107.51, 73.50) -- (108.28, 73.69) -- (109.06, 73.88) -- (109.84, 74.08) -- (110.62, 74.28) -- (111.39, 74.48) -- (112.17, 74.69) -- (112.95, 74.91) -- (113.72, 75.13) -- (114.50, 75.35) -- (115.28, 75.58) -- (116.06, 75.81) -- (116.83, 76.05) -- (117.61, 76.29) -- (118.39, 76.53) -- (119.17, 76.79) -- (119.94, 77.04) -- (120.72, 77.30) -- (121.50, 77.57) -- (122.28, 77.84) -- (123.05, 78.11) -- (123.83, 78.40) -- (124.61, 78.68) -- (125.39, 78.97) -- (126.16, 79.27) -- (126.94, 79.57) -- (127.72, 79.88) -- (128.50, 80.19) -- (129.27, 80.51) -- (130.05, 80.84) -- (130.83, 81.17) -- (131.61, 81.50) -- (132.38, 81.84) -- (133.16, 82.19) -- (133.94, 82.55) -- (134.71, 82.90) -- (135.49, 83.27) -- (136.27, 83.64) -- (137.05, 84.02) -- (137.82, 84.40) -- (138.60, 84.79) -- (139.38, 85.19) -- (140.16, 85.60) -- (140.93, 86.00) -- (141.71, 86.42) -- (142.49, 86.84) -- (143.27, 87.27) -- (144.04, 87.71) -- (144.82, 88.15) -- (145.60, 88.60) -- (146.38, 89.06) -- (147.15, 89.52) -- (147.93, 89.99) -- (148.71, 90.47) -- (149.49, 90.95) -- (150.26, 91.44) -- (151.04, 91.94) -- (151.82, 92.44) -- (152.60, 92.96) -- (153.37, 93.48) -- (154.15, 94.00) -- (154.93, 94.54) -- (155.70, 95.08) -- (156.48, 95.63) -- (157.26, 96.18) -- (158.04, 96.74) -- (158.81, 97.31) -- (159.59, 97.89) -- (160.37, 98.48) -- (161.15, 99.07) -- (161.92, 99.67) -- (162.70,100.28) -- (163.48,100.89) -- (164.26,101.51) -- (165.03,102.14) -- (165.81,102.78) -- (166.59,103.42) -- (167.37,104.07) -- (168.14,104.73) -- (168.92,105.40) -- (169.70,106.07) -- (170.48,106.76) -- (171.25,107.45) -- (172.03,108.14) -- (172.81,108.85) -- (173.59,109.56) -- (174.36,110.28) -- (175.14,111.00) -- (175.92,111.74) -- (176.69,112.48) -- (177.47,113.22) -- (178.25,113.98) -- (179.03,114.74) -- (179.80,115.51) -- (180.58,116.29) -- (181.36,117.07) -- (182.14,117.86) -- (182.91,118.66) -- (183.69,119.46) -- (184.47,120.27) -- (185.25,121.09) -- (186.02,121.92) -- (186.80,122.75) -- (187.58,123.59) -- (188.36,124.43) -- (189.13,125.28) -- (189.91,126.14) -- (190.69,127.00) -- (191.47,127.87) -- (192.24,128.75) -- (193.02,129.63) -- (193.80,130.52) -- (194.58,131.41) -- (195.35,132.31) -- (196.13,133.22) -- (196.91,134.13) -- (197.68,135.05) -- (198.46,135.97) -- (199.24,136.89) -- (200.02,137.83) -- (200.79,138.76) -- (201.57,139.70) -- (202.35,140.65) -- (203.13,141.60) -- (203.90,142.56) -- (204.68,143.52) -- (205.46,144.48) -- (206.24,145.45) -- (207.01,146.42) -- (207.79,147.39) -- (208.57,148.37) -- (209.35,149.36) -- (210.12,150.34) -- (210.90,151.33) -- (211.68,152.32) -- (212.46,153.32) -- (213.23,154.31) -- (214.01,155.31) -- (214.79,156.32) -- (215.57,157.32) -- (216.34,158.32) -- (217.12,159.33) -- (217.90,160.34) -- (218.67,161.35) -- (219.45,162.36) -- (220.23,163.38) -- (221.01,164.39) -- (221.78,165.40) -- (222.56,166.42) -- (223.34,167.43) -- (224.12,168.45) -- (224.89,169.46) -- (225.67,170.48) -- (226.45,171.49) -- (227.23,172.50) -- (228.00,173.51) -- (228.78,174.53) -- (229.56,175.53) -- (230.34,176.54) -- (231.11,177.55) -- (231.89,178.55) -- (232.67,179.55) -- (233.45,180.55) -- (234.22,181.54) -- (235.00,182.54) -- (235.78,183.53) -- (236.56,184.51) -- (237.33,185.49) -- (238.11,186.47) -- (238.89,187.45) -- (239.66,188.42) -- (240.44,189.38) -- (241.22,190.34) -- (242.00,191.30) -- (242.77,192.25) -- (243.55,193.19) -- (244.33,194.13) -- (245.11,195.06) -- (245.88,195.99) -- (246.66,196.91) -- (247.44,197.82) -- (248.22,198.73) -- (248.99,199.63) -- (249.77,200.52) -- (250.55,201.40) -- (251.33,202.28) -- (252.10,203.15) -- (252.88,204.01) -- (253.66,204.86) -- (254.44,205.70) -- (255.21,206.54) -- (255.99,207.36) -- (256.77,208.18) -- (257.55,208.98) -- (258.32,209.78) -- (259.10,210.56) -- (259.88,211.34) -- (260.65,212.11) -- (261.43,212.86) -- (262.21,213.60) -- (262.99,214.34) -- (263.76,215.06) -- (264.54,215.77) -- (265.32,216.47) -- (266.10,217.15) -- (266.87,217.83) -- (267.65,218.49) -- (268.43,219.14) -- (269.21,219.78) -- (269.98,220.40) -- (270.76,221.01) -- (271.54,221.61) -- (272.32,222.19) -- (273.09,222.76) -- (273.87,223.32) -- (274.65,223.87) -- (275.43,224.40) -- (276.20,224.91) -- (276.98,225.41) -- (277.76,225.90) -- (278.54,226.37) -- (279.31,226.83) -- (280.09,227.27) -- (280.87,227.70) -- (281.64,228.11) -- (282.42,228.51) -- (283.20,228.90) -- (283.98,229.26) -- (284.75,229.61) -- (285.53,229.95) -- (286.31,230.27) -- (287.09,230.57) -- (287.86,230.86) -- (288.64,231.13) -- (289.42,231.39) -- (290.20,231.63) -- (290.97,231.85) -- (291.75,232.06) -- (292.53,232.25) -- (293.31,232.43) -- (294.08,232.59) -- (294.86,232.73) -- (295.64,232.85) -- (296.42,232.96) -- (297.19,233.05) -- (297.97,233.13) -- (298.75,233.19) -- (299.53,233.23) -- (300.30,233.25) -- (301.08,233.26) -- (301.86,233.25) -- (302.63,233.23) -- (303.41,233.19) -- (304.19,233.13) -- (304.97,233.05) -- (305.74,232.96) -- (306.52,232.85) -- (307.30,232.73) -- (308.08,232.59) -- (308.85,232.43) -- (309.63,232.25) -- (310.41,232.06) -- (311.19,231.85) -- (311.96,231.63) -- (312.74,231.39) -- (313.52,231.13) -- (314.30,230.86) -- (315.07,230.57) -- (315.85,230.27) -- (316.63,229.95) -- (317.41,229.61) -- (318.18,229.26) -- (318.96,228.90) -- (319.74,228.51) -- (320.52,228.11) -- (321.29,227.70) -- (322.07,227.27) -- (322.85,226.83) -- (323.62,226.37) -- (324.40,225.90) -- (325.18,225.41) -- (325.96,224.91) -- (326.73,224.40) -- (327.51,223.87) -- (328.29,223.32) -- (329.07,222.76) -- (329.84,222.19) -- (330.62,221.61) -- (331.40,221.01) -- (332.18,220.40) -- (332.95,219.78) -- (333.73,219.14) -- (334.51,218.49) -- (335.29,217.83) -- (336.06,217.15) -- (336.84,216.47) -- (337.62,215.77) -- (338.40,215.06) -- (339.17,214.34) -- (339.95,213.60) -- (340.73,212.86) -- (341.51,212.11) -- (342.28,211.34) -- (343.06,210.56) -- (343.84,209.78) -- (344.61,208.98) -- (345.39,208.18) -- (346.17,207.36) -- (346.95,206.54) -- (347.72,205.70) -- (348.50,204.86) -- (349.28,204.01) -- (350.06,203.15) -- (350.83,202.28) -- (351.61,201.40) -- (352.39,200.52) -- (353.17,199.63) -- (353.94,198.73) -- (354.72,197.82) -- (355.50,196.91) -- (356.28,195.99) -- (357.05,195.06) -- (357.83,194.13) -- (358.61,193.19) -- (359.39,192.25) -- (360.16,191.30) -- (360.94,190.34) -- (361.72,189.38) -- (362.50,188.42) -- (363.27,187.45) -- (364.05,186.47) -- (364.83,185.49) -- (365.60,184.51) -- (366.38,183.53) -- (367.16,182.54) -- (367.94,181.54) -- (368.71,180.55) -- (369.49,179.55) -- (370.27,178.55) -- (371.05,177.55) -- (371.82,176.54) -- (372.60,175.53) -- (373.38,174.53) -- (374.16,173.51) -- (374.93,172.50) -- (375.71,171.49) -- (376.49,170.48) -- (377.27,169.46) -- (378.04,168.45) -- (378.82,167.43) -- (379.60,166.42) -- (380.38,165.40) -- (381.15,164.39) -- (381.93,163.38) -- (382.71,162.36) -- (383.49,161.35) -- (384.26,160.34) -- (385.04,159.33) -- (385.82,158.32) -- (386.59,157.32) -- (387.37,156.32) -- (388.15,155.31) -- (388.93,154.31) -- (389.70,153.32) -- (390.48,152.32) -- (391.26,151.33) -- (392.04,150.34) -- (392.81,149.36) -- (393.59,148.37) -- (394.37,147.39) -- (395.15,146.42) -- (395.92,145.45) -- (396.70,144.48) -- (397.48,143.52) -- (398.26,142.56) -- (399.03,141.60) -- (399.81,140.65) -- (400.59,139.70) -- (401.37,138.76) -- (402.14,137.83) -- (402.92,136.89) -- (403.70,135.97) -- (404.48,135.05) -- (405.25,134.13) -- (406.03,133.22) -- (406.81,132.31) -- (407.58,131.41) -- (408.36,130.52) -- (409.14,129.63) -- (409.92,128.75) -- (410.69,127.87) -- (411.47,127.00) -- (412.25,126.14) -- (413.03,125.28) -- (413.80,124.43) -- (414.58,123.59) -- (415.36,122.75) -- (416.14,121.92) -- (416.91,121.09) -- (417.69,120.27) -- (418.47,119.46) -- (419.25,118.66) -- (420.02,117.86) -- (420.80,117.07) -- (421.58,116.29) -- (422.36,115.51) -- (423.13,114.74) -- (423.91,113.98) -- (424.69,113.22) -- (425.47,112.48) -- (426.24,111.74) -- (427.02,111.00) -- (427.80,110.28) -- (428.57,109.56) -- (429.35,108.85) -- (430.13,108.14) -- (430.91,107.45) -- (431.68,106.76) -- (432.46,106.07) -- (433.24,105.40) -- (434.02,104.73) -- (434.79,104.07) -- (435.57,103.42) -- (436.35,102.78) -- (437.13,102.14) -- (437.90,101.51) -- (438.68,100.89) -- (439.46,100.28) -- (440.24, 99.67) -- (441.01, 99.07) -- (441.79, 98.48) -- (442.57, 97.89) -- (443.35, 97.31) -- (444.12, 96.74) -- (444.90, 96.18) -- (445.68, 95.63) -- (446.46, 95.08) -- (447.23, 94.54) -- (448.01, 94.00) -- (448.79, 93.48) -- (449.56, 92.96) -- (450.34, 92.44) -- (451.12, 91.94) -- (451.90, 91.44) -- (452.67, 90.95) -- (453.45, 90.47) -- (454.23, 89.99) -- (455.01, 89.52) -- (455.78, 89.06) -- (456.56, 88.60) -- (457.34, 88.15) -- (458.12, 87.71) -- (458.89, 87.27) -- (459.67, 86.84) -- (460.45, 86.42) -- (461.23, 86.00) -- (462.00, 85.60) -- (462.78, 85.19) -- (463.56, 84.79) -- (464.34, 84.40) -- (465.11, 84.02) -- (465.89, 83.64) -- (466.67, 83.27) -- (467.45, 82.90) -- (468.22, 82.55) -- (469.00, 82.19) -- (469.78, 81.84) -- (470.55, 81.50) -- (471.33, 81.17) -- (472.11, 80.84) -- (472.89, 80.51) -- (473.66, 80.19) -- (474.44, 79.88) -- (475.22, 79.57) -- (476.00, 79.27) -- (476.77, 78.97) -- (477.55, 78.68) -- (478.33, 78.40) -- (479.11, 78.11) -- (479.88, 77.84) -- (480.66, 77.57) -- (481.44, 77.30) -- (482.22, 77.04) -- (482.99, 76.79) -- (483.77, 76.53) -- (484.55, 76.29) -- (485.33, 76.05) -- (486.10, 75.81) -- (486.88, 75.58) -- (487.66, 75.35) -- (488.44, 75.13) -- (489.21, 74.91) -- (489.99, 74.69) -- (490.77, 74.48) -- (491.54, 74.28) -- (492.32, 74.08) -- (493.10, 73.88) -- (493.88, 73.69) -- (494.65, 73.50) -- (495.43, 73.31) -- (496.21, 73.13) -- (496.99, 72.95) -- (497.76, 72.78) -- (498.54, 72.61) -- (499.32, 72.44) -- (500.10, 72.27) -- (500.87, 72.11) -- (501.65, 71.96) -- (502.43, 71.81) -- (503.21, 71.66) -- (503.98, 71.51) -- (504.76, 71.37) -- (505.54, 71.23) -- (506.32, 71.09) -- (507.09, 70.95) -- (507.87, 70.82) -- (508.65, 70.70) -- (509.43, 70.57) -- (510.20, 70.45) -- (510.98, 70.33) -- (511.76, 70.21) -- (512.53, 70.10) -- (513.31, 69.99) -- (514.09, 69.88) -- (514.87, 69.77) -- (515.64, 69.67) -- (516.42, 69.57) -- (517.20, 69.47) -- (517.98, 69.37) -- (518.75, 69.28) -- (519.53, 69.19) -- (520.31, 69.10) -- (521.09, 69.01) -- (521.86, 68.92) -- (522.64, 68.84) -- (523.42, 68.76) -- (524.20, 68.68) -- (524.97, 68.60) -- (525.75, 68.53) -- (526.53, 68.46) -- (527.31, 68.38) -- (528.08, 68.31) -- (528.86, 68.25) -- (529.64, 68.18) -- (530.42, 68.12) -- (531.19, 68.05) -- (531.97, 67.99) -- (532.75, 67.93) -- (533.52, 67.87) -- (534.30, 67.82);
    \end{scope}

    \begin{scope}
        \path[clip] (  0.00,  0.00) rectangle (578.16,289.08);
    \end{scope}

    \begin{scope}
        \path[clip] (  0.00,  0.00) rectangle (578.16,289.08);
        \definecolor[named]{drawColor}{rgb}{0.00,0.00,0.00}

        \draw[color=drawColor,line cap=round,line join=round,fill opacity=0.00,] ( 67.86, 61.20) -- (534.30, 61.20);
        \draw[color=drawColor,line cap=round,line join=round,fill opacity=0.00,] ( 67.86, 61.20) -- ( 67.86, 55.20);
        \draw[color=drawColor,line cap=round,line join=round,fill opacity=0.00,] (145.60, 61.20) -- (145.60, 55.20);
        \draw[color=drawColor,line cap=round,line join=round,fill opacity=0.00,] (223.34, 61.20) -- (223.34, 55.20);
        \draw[color=drawColor,line cap=round,line join=round,fill opacity=0.00,] (301.08, 61.20) -- (301.08, 55.20);
        \draw[color=drawColor,line cap=round,line join=round,fill opacity=0.00,] (378.82, 61.20) -- (378.82, 55.20);
        \draw[color=drawColor,line cap=round,line join=round,fill opacity=0.00,] (456.56, 61.20) -- (456.56, 55.20);
        \draw[color=drawColor,line cap=round,line join=round,fill opacity=0.00,] (534.30, 61.20) -- (534.30, 55.20);

        \node[color=drawColor,anchor=base,inner sep=0pt, outer sep=0pt, scale=  1.00] at ( 67.86, 37.20) {-3};
        \node[color=drawColor,anchor=base,inner sep=0pt, outer sep=0pt, scale=  1.00] at (145.60, 37.20) {-2};
        \node[color=drawColor,anchor=base,inner sep=0pt, outer sep=0pt, scale=  1.00] at (223.34, 37.20) {-1};
        \node[color=drawColor,anchor=base,inner sep=0pt, outer sep=0pt, scale=  1.00] at (301.08, 37.20) {0};
        \node[color=drawColor,anchor=base,inner sep=0pt, outer sep=0pt, scale=  1.00] at (378.82, 37.20) {1};
        \node[color=drawColor,anchor=base,inner sep=0pt, outer sep=0pt, scale=  1.00] at (456.56, 37.20) {2};
        \node[color=drawColor,anchor=base,inner sep=0pt, outer sep=0pt, scale=  1.00] at (534.30, 37.20) {3};
    \end{scope}

    \begin{scope}
        \path[clip] ( 49.20, 61.20) rectangle (592.96,239.88);
        \definecolor[named]{fillColor}{rgb}{0.80,0.80,0.80}

        \draw[fill=fillColor,draw opacity=0.00,] ( 67.86, 67.82) -- ( 68.64, 67.87) -- ( 69.41, 67.93) -- ( 70.19, 67.99) -- ( 70.97, 68.05) -- ( 71.74, 68.12) -- ( 72.52, 68.18) -- ( 73.30, 68.25) -- ( 74.08, 68.31) -- ( 74.85, 68.38) -- ( 75.63, 68.46) -- ( 76.41, 68.53) -- ( 77.19, 68.60) -- ( 77.96, 68.68) -- ( 78.74, 68.76) -- ( 79.52, 68.84) -- ( 80.30, 68.92) -- ( 81.07, 69.01) -- ( 81.85, 69.10) -- ( 82.63, 69.19) -- ( 83.41, 69.28) -- ( 84.18, 69.37) -- ( 84.96, 69.47) -- ( 85.74, 69.57) -- ( 86.52, 69.67) -- ( 87.29, 69.77) -- ( 88.07, 69.88) -- ( 88.85, 69.99) -- ( 89.63, 70.10) -- ( 90.40, 70.21) -- ( 91.18, 70.33) -- ( 91.96, 70.45) -- ( 92.73, 70.57) -- ( 93.51, 70.70) -- ( 94.29, 70.82) -- ( 95.07, 70.95) -- ( 95.84, 71.09) -- ( 96.62, 71.23) -- ( 97.40, 71.37) -- ( 98.18, 71.51) -- ( 98.95, 71.66) -- ( 99.73, 71.81) -- (100.51, 71.96) -- (101.29, 72.11) -- (102.06, 72.27) -- (102.84, 72.44) -- (103.62, 72.61) -- (104.40, 72.78) -- (105.17, 72.95) -- (105.95, 73.13) -- (106.73, 73.31) -- (107.51, 73.50) -- (108.28, 73.69) -- (109.06, 73.88) -- (109.84, 74.08) -- (110.62, 74.28) -- (111.39, 74.48) -- (112.17, 74.69) -- (112.95, 74.91) -- (113.72, 75.13) -- (114.50, 75.35) -- (115.28, 75.58) -- (116.06, 75.81) -- (116.83, 76.05) -- (117.61, 76.29) -- (118.39, 76.53) -- (119.17, 76.79) -- (119.94, 77.04) -- (120.72, 77.30) -- (121.50, 77.57) -- (122.28, 77.84) -- (123.05, 78.11) -- (123.83, 78.40) -- (124.61, 78.68) -- (125.39, 78.97) -- (126.16, 79.27) -- (126.94, 79.57) -- (127.72, 79.88) -- (128.50, 80.19) -- (129.27, 80.51) -- (130.05, 80.84) -- (130.83, 81.17) -- (131.61, 81.50) -- (132.38, 81.84) -- (133.16, 82.19) -- (133.94, 82.55) -- (134.71, 82.90) -- (135.49, 83.27) -- (136.27, 83.64) -- (137.05, 84.02) -- (137.82, 84.40) -- (138.60, 84.79) -- (139.38, 85.19) -- (140.16, 85.60) -- (140.93, 86.00) -- (141.71, 86.42) -- (142.49, 86.84) -- (143.27, 87.27) -- (144.04, 87.71) -- (144.82, 88.15) -- (145.60, 88.60) -- (146.38, 89.06) -- (147.15, 89.52) -- (147.93, 89.99) -- (148.71, 90.47) -- (149.49, 90.95) -- (150.26, 91.44) -- (151.04, 91.94) -- (151.82, 92.44) -- (152.60, 92.96) -- (153.37, 93.48) -- (154.15, 94.00) -- (154.93, 94.54) -- (155.70, 95.08) -- (156.48, 95.63) -- (157.26, 96.18) -- (158.04, 96.74) -- (158.81, 97.31) -- (159.59, 97.89) -- (160.37, 98.48) -- (161.15, 99.07) -- (161.92, 99.67) -- (162.70,100.28) -- (163.48,100.89) -- (164.26,101.51) -- (165.03,102.14) -- (165.81,102.78) -- (166.59,103.42) -- (167.37,104.07) -- (168.14,104.73) -- (168.92,105.40) -- (169.70,106.07) -- (170.48,106.76) -- (171.25,107.45) -- (172.03,108.14) -- (172.81,108.85) -- (173.59,109.56) -- (174.36,110.28) -- (175.14,111.00) -- (175.92,111.74) -- (176.69,112.48) -- (177.47,113.22) -- (178.25,113.98) -- (179.03,114.74) -- (179.80,115.51) -- (180.58,116.29) -- (181.36,117.07) -- (182.14,117.86) -- (182.91,118.66) -- (183.69,119.46) -- (184.47,120.27) -- (185.25,121.09) -- (186.02,121.92) -- (186.80,122.75) -- (187.58,123.59) -- (188.36,124.43) -- (189.13,125.28) -- (189.91,126.14) -- (190.69,127.00) -- (191.47,127.87) -- (192.24,128.75) -- (193.02,129.63) -- (193.80,130.52) -- (194.58,131.41) -- (195.35,132.31) -- (196.13,133.22) -- (196.91,134.13) -- (197.68,135.05) -- (198.46,135.97) -- (199.24,136.89) -- (200.02,137.83) -- (200.79,138.76) -- (201.57,139.70) -- (202.35,140.65) -- (203.13,141.60) -- (203.90,142.56) -- (204.68,143.52) -- (205.46,144.48) -- (206.24,145.45) -- (207.01,146.42) -- (207.79,147.39) -- (208.57,148.37) -- (209.35,149.36) -- (210.12,150.34) -- (210.90,151.33) -- (211.68,152.32) -- (212.46,153.32) -- (213.23,154.31) -- (214.01,155.31) -- (214.79,156.32) -- (215.57,157.32) -- (216.34,158.32) -- (217.12,159.33) -- (217.90,160.34) -- (218.67,161.35) -- (219.45,162.36) -- (220.23,163.38) -- (221.01,164.39) -- (221.78,165.40) -- (222.56,166.42) -- (223.34,167.43) -- (224.12,168.45) -- (224.89,169.46) -- (225.67,170.48) -- (226.45,171.49) -- (227.23,172.50) -- (228.00,173.51) -- (228.78,174.53) -- (229.56,175.53) -- (230.34,176.54) -- (231.11,177.55) -- (231.89,178.55) -- (232.67,179.55) -- (233.45,180.55) -- (234.22,181.54) -- (235.00,182.54) -- (235.78,183.53) -- (236.56,184.51) -- (237.33,185.49) -- (238.11,186.47) -- (238.89,187.45) -- (239.66,188.42) -- (240.44,189.38) -- (241.22,190.34) -- (242.00,191.30) -- (242.77,192.25) -- (243.55,193.19) -- (244.33,194.13) -- (245.11,195.06) -- (245.88,195.99) -- (246.66,196.91) -- (247.44,197.82) -- (248.22,198.73) -- (248.99,199.63) -- (249.77,200.52) -- (250.55,201.40) -- (251.33,202.28) -- (252.10,203.15) -- (252.88,204.01) -- (253.66,204.86) -- (254.44,205.70) -- (255.21,206.54) -- (255.99,207.36) -- (256.77,208.18) -- (257.55,208.98) -- (258.32,209.78) -- (259.10,210.56) -- (259.88,211.34) -- (260.65,212.11) -- (261.43,212.86) -- (262.21,213.60) -- (262.99,214.34) -- (263.76,215.06) -- (264.54,215.77) -- (265.32,216.47) -- (266.10,217.15) -- (266.87,217.83) -- (267.65,218.49) -- (268.43,219.14) -- (269.21,219.78) -- (269.98,220.40) -- (270.76,221.01) -- (271.54,221.61) -- (272.32,222.19) -- (273.09,222.76) -- (273.87,223.32) -- (274.65,223.87) -- (275.43,224.40) -- (276.20,224.91) -- (276.98,225.41) -- (277.76,225.90) -- (278.54,226.37) -- (279.31,226.83) -- (280.09,227.27) -- (280.87,227.70) -- (281.64,228.11) -- (282.42,228.51) -- (283.20,228.90) -- (283.98,229.26) -- (284.75,229.61) -- (285.53,229.95) -- (286.31,230.27) -- (287.09,230.57) -- (287.86,230.86) -- (288.64,231.13) -- (289.42,231.39) -- (290.20,231.63) -- (290.97,231.85) -- (291.75,232.06) -- (292.53,232.25) -- (293.31,232.43) -- (294.08,232.59) -- (294.86,232.73) -- (295.64,232.85) -- (296.42,232.96) -- (297.19,233.05) -- (297.97,233.13) -- (298.75,233.19) -- (299.53,233.23) -- (300.30,233.25) -- (301.08,233.26) -- (301.86,233.25) -- (302.63,233.23) -- (303.41,233.19) -- (304.19,233.13) -- (304.97,233.05) -- (305.74,232.96) -- (306.52,232.85) -- (307.30,232.73) -- (308.08,232.59) -- (308.85,232.43) -- (309.63,232.25) -- (310.41,232.06) -- (311.19,231.85) -- (311.96,231.63) -- (312.74,231.39) -- (313.52,231.13) -- (314.30,230.86) -- (315.07,230.57) -- (315.85,230.27) -- (316.63,229.95) -- (317.41,229.61) -- (318.18,229.26) -- (318.96,228.90) -- (319.74,228.51) -- (320.52,228.11) -- (321.29,227.70) -- (322.07,227.27) -- (322.85,226.83) -- (323.62,226.37) -- (324.40,225.90) -- (325.18,225.41) -- (325.96,224.91) -- (326.73,224.40) -- (327.51,223.87) -- (328.29,223.32) -- (329.07,222.76) -- (329.84,222.19) -- (330.62,221.61) -- (331.40,221.01) -- (332.18,220.40) -- (332.95,219.78) -- (333.73,219.14) -- (334.51,218.49) -- (335.29,217.83) -- (336.06,217.15) -- (336.84,216.47) -- (337.62,215.77) -- (338.40,215.06) -- (339.17,214.34) -- (339.95,213.60) -- (340.73,212.86) -- (341.51,212.11) -- (342.28,211.34) -- (343.06,210.56) -- (343.84,209.78) -- (344.61,208.98) -- (345.39,208.18) -- (346.17,207.36) -- (346.95,206.54) -- (347.72,205.70) -- (348.50,204.86) -- (349.28,204.01) -- (350.06,203.15) -- (350.83,202.28) -- (351.61,201.40) -- (352.39,200.52) -- (353.17,199.63) -- (353.94,198.73) -- (354.72,197.82) -- (355.50,196.91) -- (356.28,195.99) -- (357.05,195.06) -- (357.83,194.13) -- (358.61,193.19) -- (359.39,192.25) -- (360.16,191.30) -- (360.94,190.34) -- (361.72,189.38) -- (362.50,188.42) -- (363.27,187.45) -- (364.05,186.47) -- (364.83,185.49) -- (365.60,184.51) -- (366.38,183.53) -- (367.16,182.54) -- (367.94,181.54) -- (368.71,180.55) -- (369.49,179.55) -- (370.27,178.55) -- (371.05,177.55) -- (371.82,176.54) -- (372.60,175.53) -- (373.38,174.53) -- (374.16,173.51) -- (374.93,172.50) -- (375.71,171.49) -- (376.49,170.48) -- (377.27,169.46) -- (378.04,168.45) -- (378.82,167.43) -- (378.82, 65.96) -- (378.04, 65.96) -- (377.27, 65.96) -- (376.49, 65.96) -- (375.71, 65.96) -- (374.93, 65.96) -- (374.16, 65.96) -- (373.38, 65.96) -- (372.60, 65.96) -- (371.82, 65.96) -- (371.05, 65.96) -- (370.27, 65.96) -- (369.49, 65.96) -- (368.71, 65.96) -- (367.94, 65.96) -- (367.16, 65.96) -- (366.38, 65.96) -- (365.60, 65.96) -- (364.83, 65.96) -- (364.05, 65.96) -- (363.27, 65.96) -- (362.50, 65.96) -- (361.72, 65.96) -- (360.94, 65.96) -- (360.16, 65.96) -- (359.39, 65.96) -- (358.61, 65.96) -- (357.83, 65.96) -- (357.05, 65.96) -- (356.28, 65.96) -- (355.50, 65.96) -- (354.72, 65.96) -- (353.94, 65.96) -- (353.17, 65.96) -- (352.39, 65.96) -- (351.61, 65.96) -- (350.83, 65.96) -- (350.06, 65.96) -- (349.28, 65.96) -- (348.50, 65.96) -- (347.72, 65.96) -- (346.95, 65.96) -- (346.17, 65.96) -- (345.39, 65.96) -- (344.61, 65.96) -- (343.84, 65.96) -- (343.06, 65.96) -- (342.28, 65.96) -- (341.51, 65.96) -- (340.73, 65.96) -- (339.95, 65.96) -- (339.17, 65.96) -- (338.40, 65.96) -- (337.62, 65.96) -- (336.84, 65.96) -- (336.06, 65.96) -- (335.29, 65.96) -- (334.51, 65.96) -- (333.73, 65.96) -- (332.95, 65.96) -- (332.18, 65.96) -- (331.40, 65.96) -- (330.62, 65.96) -- (329.84, 65.96) -- (329.07, 65.96) -- (328.29, 65.96) -- (327.51, 65.96) -- (326.73, 65.96) -- (325.96, 65.96) -- (325.18, 65.96) -- (324.40, 65.96) -- (323.62, 65.96) -- (322.85, 65.96) -- (322.07, 65.96) -- (321.29, 65.96) -- (320.52, 65.96) -- (319.74, 65.96) -- (318.96, 65.96) -- (318.18, 65.96) -- (317.41, 65.96) -- (316.63, 65.96) -- (315.85, 65.96) -- (315.07, 65.96) -- (314.30, 65.96) -- (313.52, 65.96) -- (312.74, 65.96) -- (311.96, 65.96) -- (311.19, 65.96) -- (310.41, 65.96) -- (309.63, 65.96) -- (308.85, 65.96) -- (308.08, 65.96) -- (307.30, 65.96) -- (306.52, 65.96) -- (305.74, 65.96) -- (304.97, 65.96) -- (304.19, 65.96) -- (303.41, 65.96) -- (302.63, 65.96) -- (301.86, 65.96) -- (301.08, 65.96) -- (300.30, 65.96) -- (299.53, 65.96) -- (298.75, 65.96) -- (297.97, 65.96) -- (297.19, 65.96) -- (296.42, 65.96) -- (295.64, 65.96) -- (294.86, 65.96) -- (294.08, 65.96) -- (293.31, 65.96) -- (292.53, 65.96) -- (291.75, 65.96) -- (290.97, 65.96) -- (290.20, 65.96) -- (289.42, 65.96) -- (288.64, 65.96) -- (287.86, 65.96) -- (287.09, 65.96) -- (286.31, 65.96) -- (285.53, 65.96) -- (284.75, 65.96) -- (283.98, 65.96) -- (283.20, 65.96) -- (282.42, 65.96) -- (281.64, 65.96) -- (280.87, 65.96) -- (280.09, 65.96) -- (279.31, 65.96) -- (278.54, 65.96) -- (277.76, 65.96) -- (276.98, 65.96) -- (276.20, 65.96) -- (275.43, 65.96) -- (274.65, 65.96) -- (273.87, 65.96) -- (273.09, 65.96) -- (272.32, 65.96) -- (271.54, 65.96) -- (270.76, 65.96) -- (269.98, 65.96) -- (269.21, 65.96) -- (268.43, 65.96) -- (267.65, 65.96) -- (266.87, 65.96) -- (266.10, 65.96) -- (265.32, 65.96) -- (264.54, 65.96) -- (263.76, 65.96) -- (262.99, 65.96) -- (262.21, 65.96) -- (261.43, 65.96) -- (260.65, 65.96) -- (259.88, 65.96) -- (259.10, 65.96) -- (258.32, 65.96) -- (257.55, 65.96) -- (256.77, 65.96) -- (255.99, 65.96) -- (255.21, 65.96) -- (254.44, 65.96) -- (253.66, 65.96) -- (252.88, 65.96) -- (252.10, 65.96) -- (251.33, 65.96) -- (250.55, 65.96) -- (249.77, 65.96) -- (248.99, 65.96) -- (248.22, 65.96) -- (247.44, 65.96) -- (246.66, 65.96) -- (245.88, 65.96) -- (245.11, 65.96) -- (244.33, 65.96) -- (243.55, 65.96) -- (242.77, 65.96) -- (242.00, 65.96) -- (241.22, 65.96) -- (240.44, 65.96) -- (239.66, 65.96) -- (238.89, 65.96) -- (238.11, 65.96) -- (237.33, 65.96) -- (236.56, 65.96) -- (235.78, 65.96) -- (235.00, 65.96) -- (234.22, 65.96) -- (233.45, 65.96) -- (232.67, 65.96) -- (231.89, 65.96) -- (231.11, 65.96) -- (230.34, 65.96) -- (229.56, 65.96) -- (228.78, 65.96) -- (228.00, 65.96) -- (227.23, 65.96) -- (226.45, 65.96) -- (225.67, 65.96) -- (224.89, 65.96) -- (224.12, 65.96) -- (223.34, 65.96) -- (222.56, 65.96) -- (221.78, 65.96) -- (221.01, 65.96) -- (220.23, 65.96) -- (219.45, 65.96) -- (218.67, 65.96) -- (217.90, 65.96) -- (217.12, 65.96) -- (216.34, 65.96) -- (215.57, 65.96) -- (214.79, 65.96) -- (214.01, 65.96) -- (213.23, 65.96) -- (212.46, 65.96) -- (211.68, 65.96) -- (210.90, 65.96) -- (210.12, 65.96) -- (209.35, 65.96) -- (208.57, 65.96) -- (207.79, 65.96) -- (207.01, 65.96) -- (206.24, 65.96) -- (205.46, 65.96) -- (204.68, 65.96) -- (203.90, 65.96) -- (203.13, 65.96) -- (202.35, 65.96) -- (201.57, 65.96) -- (200.79, 65.96) -- (200.02, 65.96) -- (199.24, 65.96) -- (198.46, 65.96) -- (197.68, 65.96) -- (196.91, 65.96) -- (196.13, 65.96) -- (195.35, 65.96) -- (194.58, 65.96) -- (193.80, 65.96) -- (193.02, 65.96) -- (192.24, 65.96) -- (191.47, 65.96) -- (190.69, 65.96) -- (189.91, 65.96) -- (189.13, 65.96) -- (188.36, 65.96) -- (187.58, 65.96) -- (186.80, 65.96) -- (186.02, 65.96) -- (185.25, 65.96) -- (184.47, 65.96) -- (183.69, 65.96) -- (182.91, 65.96) -- (182.14, 65.96) -- (181.36, 65.96) -- (180.58, 65.96) -- (179.80, 65.96) -- (179.03, 65.96) -- (178.25, 65.96) -- (177.47, 65.96) -- (176.69, 65.96) -- (175.92, 65.96) -- (175.14, 65.96) -- (174.36, 65.96) -- (173.59, 65.96) -- (172.81, 65.96) -- (172.03, 65.96) -- (171.25, 65.96) -- (170.48, 65.96) -- (169.70, 65.96) -- (168.92, 65.96) -- (168.14, 65.96) -- (167.37, 65.96) -- (166.59, 65.96) -- (165.81, 65.96) -- (165.03, 65.96) -- (164.26, 65.96) -- (163.48, 65.96) -- (162.70, 65.96) -- (161.92, 65.96) -- (161.15, 65.96) -- (160.37, 65.96) -- (159.59, 65.96) -- (158.81, 65.96) -- (158.04, 65.96) -- (157.26, 65.96) -- (156.48, 65.96) -- (155.70, 65.96) -- (154.93, 65.96) -- (154.15, 65.96) -- (153.37, 65.96) -- (152.60, 65.96) -- (151.82, 65.96) -- (151.04, 65.96) -- (150.26, 65.96) -- (149.49, 65.96) -- (148.71, 65.96) -- (147.93, 65.96) -- (147.15, 65.96) -- (146.38, 65.96) -- (145.60, 65.96) -- (144.82, 65.96) -- (144.04, 65.96) -- (143.27, 65.96) -- (142.49, 65.96) -- (141.71, 65.96) -- (140.93, 65.96) -- (140.16, 65.96) -- (139.38, 65.96) -- (138.60, 65.96) -- (137.82, 65.96) -- (137.05, 65.96) -- (136.27, 65.96) -- (135.49, 65.96) -- (134.71, 65.96) -- (133.94, 65.96) -- (133.16, 65.96) -- (132.38, 65.96) -- (131.61, 65.96) -- (130.83, 65.96) -- (130.05, 65.96) -- (129.27, 65.96) -- (128.50, 65.96) -- (127.72, 65.96) -- (126.94, 65.96) -- (126.16, 65.96) -- (125.39, 65.96) -- (124.61, 65.96) -- (123.83, 65.96) -- (123.05, 65.96) -- (122.28, 65.96) -- (121.50, 65.96) -- (120.72, 65.96) -- (119.94, 65.96) -- (119.17, 65.96) -- (118.39, 65.96) -- (117.61, 65.96) -- (116.83, 65.96) -- (116.06, 65.96) -- (115.28, 65.96) -- (114.50, 65.96) -- (113.72, 65.96) -- (112.95, 65.96) -- (112.17, 65.96) -- (111.39, 65.96) -- (110.62, 65.96) -- (109.84, 65.96) -- (109.06, 65.96) -- (108.28, 65.96) -- (107.51, 65.96) -- (106.73, 65.96) -- (105.95, 65.96) -- (105.17, 65.96) -- (104.40, 65.96) -- (103.62, 65.96) -- (102.84, 65.96) -- (102.06, 65.96) -- (101.29, 65.96) -- (100.51, 65.96) -- ( 99.73, 65.96) -- ( 98.95, 65.96) -- ( 98.18, 65.96) -- ( 97.40, 65.96) -- ( 96.62, 65.96) -- ( 95.84, 65.96) -- ( 95.07, 65.96) -- ( 94.29, 65.96) -- ( 93.51, 65.96) -- ( 92.73, 65.96) -- ( 91.96, 65.96) -- ( 91.18, 65.96) -- ( 90.40, 65.96) -- ( 89.63, 65.96) -- ( 88.85, 65.96) -- ( 88.07, 65.96) -- ( 87.29, 65.96) -- ( 86.52, 65.96) -- ( 85.74, 65.96) -- ( 84.96, 65.96) -- ( 84.18, 65.96) -- ( 83.41, 65.96) -- ( 82.63, 65.96) -- ( 81.85, 65.96) -- ( 81.07, 65.96) -- ( 80.30, 65.96) -- ( 79.52, 65.96) -- ( 78.74, 65.96) -- ( 77.96, 65.96) -- ( 77.19, 65.96) -- ( 76.41, 65.96) -- ( 75.63, 65.96) -- ( 74.85, 65.96) -- ( 74.08, 65.96) -- ( 73.30, 65.96) -- ( 72.52, 65.96) -- ( 71.74, 65.96) -- ( 70.97, 65.96) -- ( 70.19, 65.96) -- ( 69.41, 65.96) -- ( 68.64, 65.96) -- ( 67.86, 65.96) -- cycle;
        \definecolor[named]{drawColor}{rgb}{0.00,0.00,0.00}

        \draw[color=drawColor,line width= 0.8pt,line cap=round,line join=round,fill opacity=0.00,] ( 67.86, 67.82) -- ( 68.64, 67.87) -- ( 69.41, 67.93) -- ( 70.19, 67.99) -- ( 70.97, 68.05) -- ( 71.74, 68.12) -- ( 72.52, 68.18) -- ( 73.30, 68.25) -- ( 74.08, 68.31) -- ( 74.85, 68.38) -- ( 75.63, 68.46) -- ( 76.41, 68.53) -- ( 77.19, 68.60) -- ( 77.96, 68.68) -- ( 78.74, 68.76) -- ( 79.52, 68.84) -- ( 80.30, 68.92) -- ( 81.07, 69.01) -- ( 81.85, 69.10) -- ( 82.63, 69.19) -- ( 83.41, 69.28) -- ( 84.18, 69.37) -- ( 84.96, 69.47) -- ( 85.74, 69.57) -- ( 86.52, 69.67) -- ( 87.29, 69.77) -- ( 88.07, 69.88) -- ( 88.85, 69.99) -- ( 89.63, 70.10) -- ( 90.40, 70.21) -- ( 91.18, 70.33) -- ( 91.96, 70.45) -- ( 92.73, 70.57) -- ( 93.51, 70.70) -- ( 94.29, 70.82) -- ( 95.07, 70.95) -- ( 95.84, 71.09) -- ( 96.62, 71.23) -- ( 97.40, 71.37) -- ( 98.18, 71.51) -- ( 98.95, 71.66) -- ( 99.73, 71.81) -- (100.51, 71.96) -- (101.29, 72.11) -- (102.06, 72.27) -- (102.84, 72.44) -- (103.62, 72.61) -- (104.40, 72.78) -- (105.17, 72.95) -- (105.95, 73.13) -- (106.73, 73.31) -- (107.51, 73.50) -- (108.28, 73.69) -- (109.06, 73.88) -- (109.84, 74.08) -- (110.62, 74.28) -- (111.39, 74.48) -- (112.17, 74.69) -- (112.95, 74.91) -- (113.72, 75.13) -- (114.50, 75.35) -- (115.28, 75.58) -- (116.06, 75.81) -- (116.83, 76.05) -- (117.61, 76.29) -- (118.39, 76.53) -- (119.17, 76.79) -- (119.94, 77.04) -- (120.72, 77.30) -- (121.50, 77.57) -- (122.28, 77.84) -- (123.05, 78.11) -- (123.83, 78.40) -- (124.61, 78.68) -- (125.39, 78.97) -- (126.16, 79.27) -- (126.94, 79.57) -- (127.72, 79.88) -- (128.50, 80.19) -- (129.27, 80.51) -- (130.05, 80.84) -- (130.83, 81.17) -- (131.61, 81.50) -- (132.38, 81.84) -- (133.16, 82.19) -- (133.94, 82.55) -- (134.71, 82.90) -- (135.49, 83.27) -- (136.27, 83.64) -- (137.05, 84.02) -- (137.82, 84.40) -- (138.60, 84.79) -- (139.38, 85.19) -- (140.16, 85.60) -- (140.93, 86.00) -- (141.71, 86.42) -- (142.49, 86.84) -- (143.27, 87.27) -- (144.04, 87.71) -- (144.82, 88.15) -- (145.60, 88.60) -- (146.38, 89.06) -- (147.15, 89.52) -- (147.93, 89.99) -- (148.71, 90.47) -- (149.49, 90.95) -- (150.26, 91.44) -- (151.04, 91.94) -- (151.82, 92.44) -- (152.60, 92.96) -- (153.37, 93.48) -- (154.15, 94.00) -- (154.93, 94.54) -- (155.70, 95.08) -- (156.48, 95.63) -- (157.26, 96.18) -- (158.04, 96.74) -- (158.81, 97.31) -- (159.59, 97.89) -- (160.37, 98.48) -- (161.15, 99.07) -- (161.92, 99.67) -- (162.70,100.28) -- (163.48,100.89) -- (164.26,101.51) -- (165.03,102.14) -- (165.81,102.78) -- (166.59,103.42) -- (167.37,104.07) -- (168.14,104.73) -- (168.92,105.40) -- (169.70,106.07) -- (170.48,106.76) -- (171.25,107.45) -- (172.03,108.14) -- (172.81,108.85) -- (173.59,109.56) -- (174.36,110.28) -- (175.14,111.00) -- (175.92,111.74) -- (176.69,112.48) -- (177.47,113.22) -- (178.25,113.98) -- (179.03,114.74) -- (179.80,115.51) -- (180.58,116.29) -- (181.36,117.07) -- (182.14,117.86) -- (182.91,118.66) -- (183.69,119.46) -- (184.47,120.27) -- (185.25,121.09) -- (186.02,121.92) -- (186.80,122.75) -- (187.58,123.59) -- (188.36,124.43) -- (189.13,125.28) -- (189.91,126.14) -- (190.69,127.00) -- (191.47,127.87) -- (192.24,128.75) -- (193.02,129.63) -- (193.80,130.52) -- (194.58,131.41) -- (195.35,132.31) -- (196.13,133.22) -- (196.91,134.13) -- (197.68,135.05) -- (198.46,135.97) -- (199.24,136.89) -- (200.02,137.83) -- (200.79,138.76) -- (201.57,139.70) -- (202.35,140.65) -- (203.13,141.60) -- (203.90,142.56) -- (204.68,143.52) -- (205.46,144.48) -- (206.24,145.45) -- (207.01,146.42) -- (207.79,147.39) -- (208.57,148.37) -- (209.35,149.36) -- (210.12,150.34) -- (210.90,151.33) -- (211.68,152.32) -- (212.46,153.32) -- (213.23,154.31) -- (214.01,155.31) -- (214.79,156.32) -- (215.57,157.32) -- (216.34,158.32) -- (217.12,159.33) -- (217.90,160.34) -- (218.67,161.35) -- (219.45,162.36) -- (220.23,163.38) -- (221.01,164.39) -- (221.78,165.40) -- (222.56,166.42) -- (223.34,167.43) -- (224.12,168.45) -- (224.89,169.46) -- (225.67,170.48) -- (226.45,171.49) -- (227.23,172.50) -- (228.00,173.51) -- (228.78,174.53) -- (229.56,175.53) -- (230.34,176.54) -- (231.11,177.55) -- (231.89,178.55) -- (232.67,179.55) -- (233.45,180.55) -- (234.22,181.54) -- (235.00,182.54) -- (235.78,183.53) -- (236.56,184.51) -- (237.33,185.49) -- (238.11,186.47) -- (238.89,187.45) -- (239.66,188.42) -- (240.44,189.38) -- (241.22,190.34) -- (242.00,191.30) -- (242.77,192.25) -- (243.55,193.19) -- (244.33,194.13) -- (245.11,195.06) -- (245.88,195.99) -- (246.66,196.91) -- (247.44,197.82) -- (248.22,198.73) -- (248.99,199.63) -- (249.77,200.52) -- (250.55,201.40) -- (251.33,202.28) -- (252.10,203.15) -- (252.88,204.01) -- (253.66,204.86) -- (254.44,205.70) -- (255.21,206.54) -- (255.99,207.36) -- (256.77,208.18) -- (257.55,208.98) -- (258.32,209.78) -- (259.10,210.56) -- (259.88,211.34) -- (260.65,212.11) -- (261.43,212.86) -- (262.21,213.60) -- (262.99,214.34) -- (263.76,215.06) -- (264.54,215.77) -- (265.32,216.47) -- (266.10,217.15) -- (266.87,217.83) -- (267.65,218.49) -- (268.43,219.14) -- (269.21,219.78) -- (269.98,220.40) -- (270.76,221.01) -- (271.54,221.61) -- (272.32,222.19) -- (273.09,222.76) -- (273.87,223.32) -- (274.65,223.87) -- (275.43,224.40) -- (276.20,224.91) -- (276.98,225.41) -- (277.76,225.90) -- (278.54,226.37) -- (279.31,226.83) -- (280.09,227.27) -- (280.87,227.70) -- (281.64,228.11) -- (282.42,228.51) -- (283.20,228.90) -- (283.98,229.26) -- (284.75,229.61) -- (285.53,229.95) -- (286.31,230.27) -- (287.09,230.57) -- (287.86,230.86) -- (288.64,231.13) -- (289.42,231.39) -- (290.20,231.63) -- (290.97,231.85) -- (291.75,232.06) -- (292.53,232.25) -- (293.31,232.43) -- (294.08,232.59) -- (294.86,232.73) -- (295.64,232.85) -- (296.42,232.96) -- (297.19,233.05) -- (297.97,233.13) -- (298.75,233.19) -- (299.53,233.23) -- (300.30,233.25) -- (301.08,233.26) -- (301.86,233.25) -- (302.63,233.23) -- (303.41,233.19) -- (304.19,233.13) -- (304.97,233.05) -- (305.74,232.96) -- (306.52,232.85) -- (307.30,232.73) -- (308.08,232.59) -- (308.85,232.43) -- (309.63,232.25) -- (310.41,232.06) -- (311.19,231.85) -- (311.96,231.63) -- (312.74,231.39) -- (313.52,231.13) -- (314.30,230.86) -- (315.07,230.57) -- (315.85,230.27) -- (316.63,229.95) -- (317.41,229.61) -- (318.18,229.26) -- (318.96,228.90) -- (319.74,228.51) -- (320.52,228.11) -- (321.29,227.70) -- (322.07,227.27) -- (322.85,226.83) -- (323.62,226.37) -- (324.40,225.90) -- (325.18,225.41) -- (325.96,224.91) -- (326.73,224.40) -- (327.51,223.87) -- (328.29,223.32) -- (329.07,222.76) -- (329.84,222.19) -- (330.62,221.61) -- (331.40,221.01) -- (332.18,220.40) -- (332.95,219.78) -- (333.73,219.14) -- (334.51,218.49) -- (335.29,217.83) -- (336.06,217.15) -- (336.84,216.47) -- (337.62,215.77) -- (338.40,215.06) -- (339.17,214.34) -- (339.95,213.60) -- (340.73,212.86) -- (341.51,212.11) -- (342.28,211.34) -- (343.06,210.56) -- (343.84,209.78) -- (344.61,208.98) -- (345.39,208.18) -- (346.17,207.36) -- (346.95,206.54) -- (347.72,205.70) -- (348.50,204.86) -- (349.28,204.01) -- (350.06,203.15) -- (350.83,202.28) -- (351.61,201.40) -- (352.39,200.52) -- (353.17,199.63) -- (353.94,198.73) -- (354.72,197.82) -- (355.50,196.91) -- (356.28,195.99) -- (357.05,195.06) -- (357.83,194.13) -- (358.61,193.19) -- (359.39,192.25) -- (360.16,191.30) -- (360.94,190.34) -- (361.72,189.38) -- (362.50,188.42) -- (363.27,187.45) -- (364.05,186.47) -- (364.83,185.49) -- (365.60,184.51) -- (366.38,183.53) -- (367.16,182.54) -- (367.94,181.54) -- (368.71,180.55) -- (369.49,179.55) -- (370.27,178.55) -- (371.05,177.55) -- (371.82,176.54) -- (372.60,175.53) -- (373.38,174.53) -- (374.16,173.51) -- (374.93,172.50) -- (375.71,171.49) -- (376.49,170.48) -- (377.27,169.46) -- (378.04,168.45) -- (378.82,167.43) -- (379.60,166.42) -- (380.38,165.40) -- (381.15,164.39) -- (381.93,163.38) -- (382.71,162.36) -- (383.49,161.35) -- (384.26,160.34) -- (385.04,159.33) -- (385.82,158.32) -- (386.59,157.32) -- (387.37,156.32) -- (388.15,155.31) -- (388.93,154.31) -- (389.70,153.32) -- (390.48,152.32) -- (391.26,151.33) -- (392.04,150.34) -- (392.81,149.36) -- (393.59,148.37) -- (394.37,147.39) -- (395.15,146.42) -- (395.92,145.45) -- (396.70,144.48) -- (397.48,143.52) -- (398.26,142.56) -- (399.03,141.60) -- (399.81,140.65) -- (400.59,139.70) -- (401.37,138.76) -- (402.14,137.83) -- (402.92,136.89) -- (403.70,135.97) -- (404.48,135.05) -- (405.25,134.13) -- (406.03,133.22) -- (406.81,132.31) -- (407.58,131.41) -- (408.36,130.52) -- (409.14,129.63) -- (409.92,128.75) -- (410.69,127.87) -- (411.47,127.00) -- (412.25,126.14) -- (413.03,125.28) -- (413.80,124.43) -- (414.58,123.59) -- (415.36,122.75) -- (416.14,121.92) -- (416.91,121.09) -- (417.69,120.27) -- (418.47,119.46) -- (419.25,118.66) -- (420.02,117.86) -- (420.80,117.07) -- (421.58,116.29) -- (422.36,115.51) -- (423.13,114.74) -- (423.91,113.98) -- (424.69,113.22) -- (425.47,112.48) -- (426.24,111.74) -- (427.02,111.00) -- (427.80,110.28) -- (428.57,109.56) -- (429.35,108.85) -- (430.13,108.14) -- (430.91,107.45) -- (431.68,106.76) -- (432.46,106.07) -- (433.24,105.40) -- (434.02,104.73) -- (434.79,104.07) -- (435.57,103.42) -- (436.35,102.78) -- (437.13,102.14) -- (437.90,101.51) -- (438.68,100.89) -- (439.46,100.28) -- (440.24, 99.67) -- (441.01, 99.07) -- (441.79, 98.48) -- (442.57, 97.89) -- (443.35, 97.31) -- (444.12, 96.74) -- (444.90, 96.18) -- (445.68, 95.63) -- (446.46, 95.08) -- (447.23, 94.54) -- (448.01, 94.00) -- (448.79, 93.48) -- (449.56, 92.96) -- (450.34, 92.44) -- (451.12, 91.94) -- (451.90, 91.44) -- (452.67, 90.95) -- (453.45, 90.47) -- (454.23, 89.99) -- (455.01, 89.52) -- (455.78, 89.06) -- (456.56, 88.60) -- (457.34, 88.15) -- (458.12, 87.71) -- (458.89, 87.27) -- (459.67, 86.84) -- (460.45, 86.42) -- (461.23, 86.00) -- (462.00, 85.60) -- (462.78, 85.19) -- (463.56, 84.79) -- (464.34, 84.40) -- (465.11, 84.02) -- (465.89, 83.64) -- (466.67, 83.27) -- (467.45, 82.90) -- (468.22, 82.55) -- (469.00, 82.19) -- (469.78, 81.84) -- (470.55, 81.50) -- (471.33, 81.17) -- (472.11, 80.84) -- (472.89, 80.51) -- (473.66, 80.19) -- (474.44, 79.88) -- (475.22, 79.57) -- (476.00, 79.27) -- (476.77, 78.97) -- (477.55, 78.68) -- (478.33, 78.40) -- (479.11, 78.11) -- (479.88, 77.84) -- (480.66, 77.57) -- (481.44, 77.30) -- (482.22, 77.04) -- (482.99, 76.79) -- (483.77, 76.53) -- (484.55, 76.29) -- (485.33, 76.05) -- (486.10, 75.81) -- (486.88, 75.58) -- (487.66, 75.35) -- (488.44, 75.13) -- (489.21, 74.91) -- (489.99, 74.69) -- (490.77, 74.48) -- (491.54, 74.28) -- (492.32, 74.08) -- (493.10, 73.88) -- (493.88, 73.69) -- (494.65, 73.50) -- (495.43, 73.31) -- (496.21, 73.13) -- (496.99, 72.95) -- (497.76, 72.78) -- (498.54, 72.61) -- (499.32, 72.44) -- (500.10, 72.27) -- (500.87, 72.11) -- (501.65, 71.96) -- (502.43, 71.81) -- (503.21, 71.66) -- (503.98, 71.51) -- (504.76, 71.37) -- (505.54, 71.23) -- (506.32, 71.09) -- (507.09, 70.95) -- (507.87, 70.82) -- (508.65, 70.70) -- (509.43, 70.57) -- (510.20, 70.45) -- (510.98, 70.33) -- (511.76, 70.21) -- (512.53, 70.10) -- (513.31, 69.99) -- (514.09, 69.88) -- (514.87, 69.77) -- (515.64, 69.67) -- (516.42, 69.57) -- (517.20, 69.47) -- (517.98, 69.37) -- (518.75, 69.28) -- (519.53, 69.19) -- (520.31, 69.10) -- (521.09, 69.01) -- (521.86, 68.92) -- (522.64, 68.84) -- (523.42, 68.76) -- (524.20, 68.68) -- (524.97, 68.60) -- (525.75, 68.53) -- (526.53, 68.46) -- (527.31, 68.38) -- (528.08, 68.31) -- (528.86, 68.25) -- (529.64, 68.18) -- (530.42, 68.12) -- (531.19, 68.05) -- (531.97, 67.99) -- (532.75, 67.93) -- (533.52, 67.87) -- (534.30, 67.82);
        \definecolor[named]{drawColor}{rgb}{0.00,0.00,0.00}

        \node[color=drawColor,anchor=base,inner sep=0pt, outer sep=0pt, scale=  1.20] at (477.13,203.23) {$P(X\leq x)=\displaystyle{\int_{-\infty}^x \frac{1}{\sqrt{2\pi}}e^{-\frac{t^2}{2}} dt}$};
    \end{scope}
\end{tikzpicture}


	\centering
	{\large\begin{tabular}{r|rrrrrrrrrr}
  \hline
  & 0.00 & 0.01 & 0.02 & 0.03 & 0.04 & 0.05 & 0.06 & 0.07 & 0.08 & 0.09 \\ 
  \hline
  0.0 & 0.5000 & 0.5040 & 0.5080 & 0.5120 & 0.5160 & 0.5199 & 0.5239 & 0.5279 & 0.5319 & 0.5359 \\ 
  0.1 & 0.5398 & 0.5438 & 0.5478 & 0.5517 & 0.5557 & 0.5596 & 0.5636 & 0.5675 & 0.5714 & 0.5753 \\ 
  0.2 & 0.5793 & 0.5832 & 0.5871 & 0.5910 & 0.5948 & 0.5987 & 0.6026 & 0.6064 & 0.6103 & 0.6141 \\ 
  0.3 & 0.6179 & 0.6217 & 0.6255 & 0.6293 & 0.6331 & 0.6368 & 0.6406 & 0.6443 & 0.6480 & 0.6517 \\ 
  0.4 & 0.6554 & 0.6591 & 0.6628 & 0.6664 & 0.6700 & 0.6736 & 0.6772 & 0.6808 & 0.6844 & 0.6879 \\ 
  0.5 & 0.6915 & 0.6950 & 0.6985 & 0.7019 & 0.7054 & 0.7088 & 0.7123 & 0.7157 & 0.7190 & 0.7224 \\ 
  0.6 & 0.7257 & 0.7291 & 0.7324 & 0.7357 & 0.7389 & 0.7422 & 0.7454 & 0.7486 & 0.7517 & 0.7549 \\ 
  0.7 & 0.7580 & 0.7611 & 0.7642 & 0.7673 & 0.7704 & 0.7734 & 0.7764 & 0.7794 & 0.7823 & 0.7852 \\ 
  0.8 & 0.7881 & 0.7910 & 0.7939 & 0.7967 & 0.7995 & 0.8023 & 0.8051 & 0.8078 & 0.8106 & 0.8133 \\ 
  0.9 & 0.8159 & 0.8186 & 0.8212 & 0.8238 & 0.8264 & 0.8289 & 0.8315 & 0.8340 & 0.8365 & 0.8389 \\ 
  1.0 & 0.8413 & 0.8438 & 0.8461 & 0.8485 & 0.8508 & 0.8531 & 0.8554 & 0.8577 & 0.8599 & 0.8621 \\ 
  1.1 & 0.8643 & 0.8665 & 0.8686 & 0.8708 & 0.8729 & 0.8749 & 0.8770 & 0.8790 & 0.8810 & 0.8830 \\ 
  1.2 & 0.8849 & 0.8869 & 0.8888 & 0.8907 & 0.8925 & 0.8944 & 0.8962 & 0.8980 & 0.8997 & 0.9015 \\ 
  1.3 & 0.9032 & 0.9049 & 0.9066 & 0.9082 & 0.9099 & 0.9115 & 0.9131 & 0.9147 & 0.9162 & 0.9177 \\ 
  1.4 & 0.9192 & 0.9207 & 0.9222 & 0.9236 & 0.9251 & 0.9265 & 0.9279 & 0.9292 & 0.9306 & 0.9319 \\ 
  1.5 & 0.9332 & 0.9345 & 0.9357 & 0.9370 & 0.9382 & 0.9394 & 0.9406 & 0.9418 & 0.9429 & 0.9441 \\ 
  1.6 & 0.9452 & 0.9463 & 0.9474 & 0.9484 & 0.9495 & 0.9505 & 0.9515 & 0.9525 & 0.9535 & 0.9545 \\ 
  1.7 & 0.9554 & 0.9564 & 0.9573 & 0.9582 & 0.9591 & 0.9599 & 0.9608 & 0.9616 & 0.9625 & 0.9633 \\ 
  1.8 & 0.9641 & 0.9649 & 0.9656 & 0.9664 & 0.9671 & 0.9678 & 0.9686 & 0.9693 & 0.9699 & 0.9706 \\ 
  1.9 & 0.9713 & 0.9719 & 0.9726 & 0.9732 & 0.9738 & 0.9744 & 0.9750 & 0.9756 & 0.9761 & 0.9767 \\ 
  2.0 & 0.9772 & 0.9778 & 0.9783 & 0.9788 & 0.9793 & 0.9798 & 0.9803 & 0.9808 & 0.9812 & 0.9817 \\ 
  2.1 & 0.9821 & 0.9826 & 0.9830 & 0.9834 & 0.9838 & 0.9842 & 0.9846 & 0.9850 & 0.9854 & 0.9857 \\ 
  2.2 & 0.9861 & 0.9864 & 0.9868 & 0.9871 & 0.9875 & 0.9878 & 0.9881 & 0.9884 & 0.9887 & 0.9890 \\ 
  2.3 & 0.9893 & 0.9896 & 0.9898 & 0.9901 & 0.9904 & 0.9906 & 0.9909 & 0.9911 & 0.9913 & 0.9916 \\ 
  2.4 & 0.9918 & 0.9920 & 0.9922 & 0.9925 & 0.9927 & 0.9929 & 0.9931 & 0.9932 & 0.9934 & 0.9936 \\ 
  2.5 & 0.9938 & 0.9940 & 0.9941 & 0.9943 & 0.9945 & 0.9946 & 0.9948 & 0.9949 & 0.9951 & 0.9952 \\ 
  2.6 & 0.9953 & 0.9955 & 0.9956 & 0.9957 & 0.9959 & 0.9960 & 0.9961 & 0.9962 & 0.9963 & 0.9964 \\ 
  2.7 & 0.9965 & 0.9966 & 0.9967 & 0.9968 & 0.9969 & 0.9970 & 0.9971 & 0.9972 & 0.9973 & 0.9974 \\ 
  2.8 & 0.9974 & 0.9975 & 0.9976 & 0.9977 & 0.9977 & 0.9978 & 0.9979 & 0.9979 & 0.9980 & 0.9981 \\ 
  2.9 & 0.9981 & 0.9982 & 0.9982 & 0.9983 & 0.9984 & 0.9984 & 0.9985 & 0.9985 & 0.9986 & 0.9986 \\ 
  3.0 & 0.9987 & 0.9987 & 0.9987 & 0.9988 & 0.9988 & 0.9989 & 0.9989 & 0.9989 & 0.9990 & 0.9990 \\ 
  \hline
\end{tabular}
}
\end{landscape}

\end{document}
